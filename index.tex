% Options for packages loaded elsewhere
\PassOptionsToPackage{unicode}{hyperref}
\PassOptionsToPackage{hyphens}{url}
\PassOptionsToPackage{dvipsnames,svgnames,x11names}{xcolor}
%
\documentclass[
  b5paperpaper,
  DIV=11,
  numbers=noendperiod]{scrreprt}

\usepackage{amsmath,amssymb}
\usepackage{lmodern}
\usepackage{iftex}
\ifPDFTeX
  \usepackage[T1]{fontenc}
  \usepackage[utf8]{inputenc}
  \usepackage{textcomp} % provide euro and other symbols
\else % if luatex or xetex
  \usepackage{unicode-math}
  \defaultfontfeatures{Scale=MatchLowercase}
  \defaultfontfeatures[\rmfamily]{Ligatures=TeX,Scale=1}
  \setmainfont[]{NanumBarunPen}
  \setsansfont[]{NanumGothic}
  \setmonofont[]{D2Coding}
\fi
% Use upquote if available, for straight quotes in verbatim environments
\IfFileExists{upquote.sty}{\usepackage{upquote}}{}
\IfFileExists{microtype.sty}{% use microtype if available
  \usepackage[]{microtype}
  \UseMicrotypeSet[protrusion]{basicmath} % disable protrusion for tt fonts
}{}
\makeatletter
\@ifundefined{KOMAClassName}{% if non-KOMA class
  \IfFileExists{parskip.sty}{%
    \usepackage{parskip}
  }{% else
    \setlength{\parindent}{0pt}
    \setlength{\parskip}{6pt plus 2pt minus 1pt}}
}{% if KOMA class
  \KOMAoptions{parskip=half}}
\makeatother
\usepackage{xcolor}
\usepackage[margin=2cm]{geometry}
\setlength{\emergencystretch}{3em} % prevent overfull lines
\setcounter{secnumdepth}{5}
% Make \paragraph and \subparagraph free-standing
\ifx\paragraph\undefined\else
  \let\oldparagraph\paragraph
  \renewcommand{\paragraph}[1]{\oldparagraph{#1}\mbox{}}
\fi
\ifx\subparagraph\undefined\else
  \let\oldsubparagraph\subparagraph
  \renewcommand{\subparagraph}[1]{\oldsubparagraph{#1}\mbox{}}
\fi

\usepackage{color}
\usepackage{fancyvrb}
\newcommand{\VerbBar}{|}
\newcommand{\VERB}{\Verb[commandchars=\\\{\}]}
\DefineVerbatimEnvironment{Highlighting}{Verbatim}{commandchars=\\\{\}}
% Add ',fontsize=\small' for more characters per line
\usepackage{framed}
\definecolor{shadecolor}{RGB}{241,243,245}
\newenvironment{Shaded}{\begin{snugshade}}{\end{snugshade}}
\newcommand{\AlertTok}[1]{\textcolor[rgb]{0.68,0.00,0.00}{#1}}
\newcommand{\AnnotationTok}[1]{\textcolor[rgb]{0.37,0.37,0.37}{#1}}
\newcommand{\AttributeTok}[1]{\textcolor[rgb]{0.40,0.45,0.13}{#1}}
\newcommand{\BaseNTok}[1]{\textcolor[rgb]{0.68,0.00,0.00}{#1}}
\newcommand{\BuiltInTok}[1]{\textcolor[rgb]{0.00,0.23,0.31}{#1}}
\newcommand{\CharTok}[1]{\textcolor[rgb]{0.13,0.47,0.30}{#1}}
\newcommand{\CommentTok}[1]{\textcolor[rgb]{0.37,0.37,0.37}{#1}}
\newcommand{\CommentVarTok}[1]{\textcolor[rgb]{0.37,0.37,0.37}{\textit{#1}}}
\newcommand{\ConstantTok}[1]{\textcolor[rgb]{0.56,0.35,0.01}{#1}}
\newcommand{\ControlFlowTok}[1]{\textcolor[rgb]{0.00,0.23,0.31}{#1}}
\newcommand{\DataTypeTok}[1]{\textcolor[rgb]{0.68,0.00,0.00}{#1}}
\newcommand{\DecValTok}[1]{\textcolor[rgb]{0.68,0.00,0.00}{#1}}
\newcommand{\DocumentationTok}[1]{\textcolor[rgb]{0.37,0.37,0.37}{\textit{#1}}}
\newcommand{\ErrorTok}[1]{\textcolor[rgb]{0.68,0.00,0.00}{#1}}
\newcommand{\ExtensionTok}[1]{\textcolor[rgb]{0.00,0.23,0.31}{#1}}
\newcommand{\FloatTok}[1]{\textcolor[rgb]{0.68,0.00,0.00}{#1}}
\newcommand{\FunctionTok}[1]{\textcolor[rgb]{0.28,0.35,0.67}{#1}}
\newcommand{\ImportTok}[1]{\textcolor[rgb]{0.00,0.46,0.62}{#1}}
\newcommand{\InformationTok}[1]{\textcolor[rgb]{0.37,0.37,0.37}{#1}}
\newcommand{\KeywordTok}[1]{\textcolor[rgb]{0.00,0.23,0.31}{#1}}
\newcommand{\NormalTok}[1]{\textcolor[rgb]{0.00,0.23,0.31}{#1}}
\newcommand{\OperatorTok}[1]{\textcolor[rgb]{0.37,0.37,0.37}{#1}}
\newcommand{\OtherTok}[1]{\textcolor[rgb]{0.00,0.23,0.31}{#1}}
\newcommand{\PreprocessorTok}[1]{\textcolor[rgb]{0.68,0.00,0.00}{#1}}
\newcommand{\RegionMarkerTok}[1]{\textcolor[rgb]{0.00,0.23,0.31}{#1}}
\newcommand{\SpecialCharTok}[1]{\textcolor[rgb]{0.37,0.37,0.37}{#1}}
\newcommand{\SpecialStringTok}[1]{\textcolor[rgb]{0.13,0.47,0.30}{#1}}
\newcommand{\StringTok}[1]{\textcolor[rgb]{0.13,0.47,0.30}{#1}}
\newcommand{\VariableTok}[1]{\textcolor[rgb]{0.07,0.07,0.07}{#1}}
\newcommand{\VerbatimStringTok}[1]{\textcolor[rgb]{0.13,0.47,0.30}{#1}}
\newcommand{\WarningTok}[1]{\textcolor[rgb]{0.37,0.37,0.37}{\textit{#1}}}

\providecommand{\tightlist}{%
  \setlength{\itemsep}{0pt}\setlength{\parskip}{0pt}}\usepackage{longtable,booktabs,array}
\usepackage{calc} % for calculating minipage widths
% Correct order of tables after \paragraph or \subparagraph
\usepackage{etoolbox}
\makeatletter
\patchcmd\longtable{\par}{\if@noskipsec\mbox{}\fi\par}{}{}
\makeatother
% Allow footnotes in longtable head/foot
\IfFileExists{footnotehyper.sty}{\usepackage{footnotehyper}}{\usepackage{footnote}}
\makesavenoteenv{longtable}
\usepackage{graphicx}
\makeatletter
\def\maxwidth{\ifdim\Gin@nat@width>\linewidth\linewidth\else\Gin@nat@width\fi}
\def\maxheight{\ifdim\Gin@nat@height>\textheight\textheight\else\Gin@nat@height\fi}
\makeatother
% Scale images if necessary, so that they will not overflow the page
% margins by default, and it is still possible to overwrite the defaults
% using explicit options in \includegraphics[width, height, ...]{}
\setkeys{Gin}{width=\maxwidth,height=\maxheight,keepaspectratio}
% Set default figure placement to htbp
\makeatletter
\def\fps@figure{htbp}
\makeatother

\KOMAoption{captions}{tableheading}
\usepackage{lastpage}
\usepackage{fancyhdr}
\definecolor{col1}{RGB}{5, 53, 255}
\fancypagestyle{ttlpage}{\fancyfoot[C]{{\thepage} of \pageref{LastPage}}}
\thispagestyle{ttlpage}
\pagestyle{fancy}
\fancyhead[EL,OL]{\rightmark}
\fancyhead[EC,OC]{}
\fancyhead[ER,OR]{빛에듀(Bit-Edu)}
\fancyfoot[EL,OL]{\color{col1}{한국R사용자회}}
\fancyfoot[EC,OC]{\textbf{리보그 세상(Reeborg's World)}}
\fancyfoot[ER,OR] {총 \pageref{LastPage} 중 {\thepage}}
\makeatletter
\makeatother
\makeatletter
\@ifpackageloaded{bookmark}{}{\usepackage{bookmark}}
\makeatother
\makeatletter
\@ifpackageloaded{caption}{}{\usepackage{caption}}
\AtBeginDocument{%
\ifdefined\contentsname
  \renewcommand*\contentsname{목차}
\else
  \newcommand\contentsname{목차}
\fi
\ifdefined\listfigurename
  \renewcommand*\listfigurename{그림 목록}
\else
  \newcommand\listfigurename{그림 목록}
\fi
\ifdefined\listtablename
  \renewcommand*\listtablename{표 목록}
\else
  \newcommand\listtablename{표 목록}
\fi
\ifdefined\figurename
  \renewcommand*\figurename{그림}
\else
  \newcommand\figurename{그림}
\fi
\ifdefined\tablename
  \renewcommand*\tablename{표}
\else
  \newcommand\tablename{표}
\fi
}
\@ifpackageloaded{float}{}{\usepackage{float}}
\floatstyle{ruled}
\@ifundefined{c@chapter}{\newfloat{codelisting}{h}{lop}}{\newfloat{codelisting}{h}{lop}[chapter]}
\floatname{codelisting}{목록}
\newcommand*\listoflistings{\listof{codelisting}{코드 목록}}
\makeatother
\makeatletter
\@ifpackageloaded{caption}{}{\usepackage{caption}}
\@ifpackageloaded{subcaption}{}{\usepackage{subcaption}}
\makeatother
\makeatletter
\@ifpackageloaded{tcolorbox}{}{\usepackage[many]{tcolorbox}}
\makeatother
\makeatletter
\@ifundefined{shadecolor}{\definecolor{shadecolor}{rgb}{.97, .97, .97}}
\makeatother
\makeatletter
\makeatother
\ifLuaTeX
\usepackage[bidi=basic]{babel}
\else
\usepackage[bidi=default]{babel}
\fi
% get rid of language-specific shorthands (see #6817):
\let\LanguageShortHands\languageshorthands
\def\languageshorthands#1{}
\ifLuaTeX
  \usepackage{selnolig}  % disable illegal ligatures
\fi
\IfFileExists{bookmark.sty}{\usepackage{bookmark}}{\usepackage{hyperref}}
\IfFileExists{xurl.sty}{\usepackage{xurl}}{} % add URL line breaks if available
\urlstyle{same} % disable monospaced font for URLs
\hypersetup{
  pdftitle={리보그 세상},
  pdfauthor={한국 R 사용자회},
  pdflang={kr},
  colorlinks=true,
  linkcolor={blue},
  filecolor={Maroon},
  citecolor={Blue},
  urlcolor={Blue},
  pdfcreator={LaTeX via pandoc}}

\title{리보그 세상}
\usepackage{etoolbox}
\makeatletter
\providecommand{\subtitle}[1]{% add subtitle to \maketitle
  \apptocmd{\@title}{\par {\large #1 \par}}{}{}
}
\makeatother
\subtitle{블록 프로그래밍 언어에서 R/파이썬으로 가는 중간 언어}
\author{한국 R 사용자회}
\date{2022년 11월 10일}

\begin{document}
\maketitle
\ifdefined\Shaded\renewenvironment{Shaded}{\begin{tcolorbox}[interior hidden, sharp corners, boxrule=0pt, borderline west={3pt}{0pt}{shadecolor}, enhanced, breakable, frame hidden]}{\end{tcolorbox}}\fi

\renewcommand*\contentsname{목차}
{
\hypersetup{linkcolor=}
\setcounter{tocdepth}{2}
\tableofcontents
}
\bookmarksetup{startatroot}

\hypertarget{reeborg}{%
\chapter*{리보그 세상}\label{reeborg}}
\addcontentsline{toc}{chapter}{리보그 세상}

\markboth{리보그 세상}{리보그 세상}

\textbf{리보그 세상(Reeborg's World)}는
\href{https://statkclee.github.io/unplugged/}{컴퓨터 과학 언플러그드}를
통해 전혀 컴퓨터가 없는 상태에서 주요 개념을 빠른 시간내에 잡을 수 있고,
이후 블록 프로그래밍(Block Programming)으로 MIT에서 개발한
\href{https://scratch.mit.edu/}{스크래치(Scratch)}와
\href{https://tidyblocks.tech/}{타이디블록(Tidyblocks)}를 통해 비쥬얼
프로그래밍을 즐길 수 있다. 그 다음 단계로 소프트웨어 프로그래밍 언어인
파이썬, 데이터 프로그래밍 언어인 R 로 바로 넘어가는 것이 아니라
\texttt{move()}, \texttt{turn\_left()} 두개의 동사를 활용한
\href{https://reeborg.ca/}{리보그 세상(Reeborg's World)}로 기초를 탄탄히
다져놓아야 한다. \href{https://reeborg.ca/docs/ko/}{리보그 세상
도움말}도 본격적인 프로그래밍으로 넘어가는데 도움이 될 것이다. 리보그는
저자가 해답(Solution)을 특별히 제시하고 있지는 않고 힌트를 제시하는
수준으로 인터넷에 정보가 공유되고 있으니 각자 리보그 과제를 함께
고민하면서 생각근육을 많이 키워나갔으면 합니다.

\includegraphics[width=6.65625in,height=\textheight]{./fig/reeborg-project.png}

\hypertarget{rur-ple}{%
\section*{러플}\label{rur-ple}}
\addcontentsline{toc}{section}{러플}

\markright{러플}

\href{https://reeborg.ca/}{리보그 세상(Reeborg's World)}은
\href{https://statkclee.github.io/rur-ple/}{러플(RUR-PLE)}으로 André
Roberge 박사가 2004년부터 2010년까지 PC 설치형 버젼 코딩 교육을 제작한
것으로 인기를 얻었고 한글 버젼도
\href{https://statkclee.github.io/rur-ple/}{러플(RUR-PLE)} 웹사이트에서
확인 가능하다.

\hypertarget{code-persepctive}{%
\section*{프로그래밍과 문제해결}\label{code-persepctive}}
\addcontentsline{toc}{section}{프로그래밍과 문제해결}

\markright{프로그래밍과 문제해결}

미국 항공우주국(NASA) JPL(제트 추진 연구소) 연구원 안드레 카스타노가
미국 남가주 방과후 교육에 수년에 걸쳐 가다듬은 소프트웨어 및 문제해결
교육 과정을 제작하여 인터넷에 공개하였는
\href{https://reeborg.ca/}{리보그 세상(Reeborg's World)} 초기 버젼을
기반으로 제작하였다.
\href{https://statkclee.github.io/code-perspectives/}{프로그래밍과
문제해결} 웹사이트를 참고한다.

\hypertarget{reeborg-programming}{%
\section*{리보그 기본 코딩방법}\label{reeborg-programming}}
\addcontentsline{toc}{section}{리보그 기본 코딩방법}

\markright{리보그 기본 코딩방법}

\href{https://scratch.mit.edu/}{스크래치(Scratch)}와
\href{https://tidyblocks.tech/}{타이디블록(Tidyblocks)} 같은 블록
프로그래밍 언어가 WIMP를 활용한 클릭(click), 드래그(drag), 드랍(drop)
방식이라면 리보그는 별도 키보드가 있어 이를 활용하여 키보드와 편집기를
이용한 본격적인 프로그래밍 이전에 유용하게 활용할 수 있다.

\includegraphics[width=6.84375in,height=\textheight]{./fig/reeborg_howto.gif}

\hypertarget{useful-website}{%
\section*{유용한 코딩 웹사이트}\label{useful-website}}
\addcontentsline{toc}{section}{유용한 코딩 웹사이트}

\markright{유용한 코딩 웹사이트}

\begin{itemize}
\item
  \href{https://statkclee.github.io/unplugged/}{컴퓨터 과학 언플러그드}
\item
  \href{https://tidyblocks.tech/}{TidyBlocks}
\item
  \href{https://reeborg.ca/docs/ko/}{리보그 세상 도움말}
\item
  \href{https://statkclee.github.io/pythonlearn-kr/}{모두를 위한 파이썬}
\item
  \href{http://statkclee.github.io/swcarpentry-version-5-3-new/}{소프트웨어
  카펜트리}
\item
  \href{http://statkclee.github.io/data-science/}{데이터 과학}
\item
  \textbf{글쓰기 저작}

  \begin{itemize}
  \tightlist
  \item
    \href{https://statkclee.github.io/comp_document/}{컴퓨터 문서저작}
  \item
    \href{https://statkclee.github.io/ds-authoring/}{데이터 과학
    기고문과 발표자료 사례}
  \end{itemize}
\item
  \href{https://use-r.kr/}{한국 R 컨퍼런스}
\end{itemize}

\part{직선 1차원}

\hypertarget{intro}{%
\chapter{집 (Home)}\label{intro}}

윈도우의 경우 \texttt{컨트롤+F5} 단축키를 누르게 되면 웹페이지를 열 수
있다. 이를 통해 리보그의 해당 과제을 파악할 수 있고 목적을 명확히
인식하고 문제를 해결해나갈 수 있다.

\hypertarget{alone}{%
\section{Alone}\label{alone}}

\begin{itemize}
\tightlist
\item
  \href{https://reeborg.ca/reeborg.html?lang=ko-en\&mode=python\&menu=worlds\%2Fmenus\%2Freeborg_intro_en.json\&name=Alone\&url=worlds\%2Ftutorial_en\%2Falone.json}{문제
  바로가기}
\item
  '리보그의 키보드'의 각종 명령어 사용법을 익힌다.
\end{itemize}

\hypertarget{uxc2e4uxd589uxacb0uxacfc}{%
\subsection{실행결과}\label{uxc2e4uxd589uxacb0uxacfc}}

\begin{Shaded}
\begin{Highlighting}[]
\ControlFlowTok{if}\NormalTok{ (knitr}\SpecialCharTok{:::}\FunctionTok{is\_latex\_output}\NormalTok{()) \{}
\NormalTok{  knitr}\SpecialCharTok{::}\FunctionTok{asis\_output}\NormalTok{(}\StringTok{\textquotesingle{}}\SpecialCharTok{\textbackslash{}\textbackslash{}}\StringTok{url\{....\}\textquotesingle{}}\NormalTok{)}
\NormalTok{\} }\ControlFlowTok{else}\NormalTok{ \{}
\NormalTok{  knitr}\SpecialCharTok{::}\FunctionTok{include\_graphics}\NormalTok{(}\StringTok{"fig/Alone.gif"}\NormalTok{)}
\NormalTok{\}}
\end{Highlighting}
\end{Shaded}

\url{....}

\hypertarget{uxcf54uxb4dc}{%
\subsection{코드}\label{uxcf54uxb4dc}}

\begin{Shaded}
\begin{Highlighting}[]
\FunctionTok{move}\NormalTok{()}
\FunctionTok{move}\NormalTok{()}
\FunctionTok{turn\_left}\NormalTok{()}
\FunctionTok{move}\NormalTok{()}
\end{Highlighting}
\end{Shaded}

\hypertarget{home-01}{%
\section{Home 1}\label{home-01}}

\begin{itemize}
\tightlist
\item
  \href{https://reeborg.ca/reeborg.html?lang=ko-en\&mode=python\&menu=worlds\%2Fmenus\%2Freeborg_intro_en.json\&name=Home\%201\&url=worlds\%2Ftutorial_en\%2Fhome1.json}{문제
  바로가기}
\item
  리보그가 집을 떠나 있어 집으로 되돌아간다.

  \begin{itemize}
  \tightlist
  \item
    로봇의 좌표 (x, y) = (3, 1) 라서 로봇의 최종위치를 반드시 (x, y) =
    (1, 1) 으로 이동시킨다.
  \end{itemize}
\end{itemize}

\hypertarget{uxc2e4uxd589uxacb0uxacfc-1}{%
\subsection{실행결과}\label{uxc2e4uxd589uxacb0uxacfc-1}}

\begin{Shaded}
\begin{Highlighting}[]
\ControlFlowTok{if}\NormalTok{ (knitr}\SpecialCharTok{:::}\FunctionTok{is\_latex\_output}\NormalTok{()) \{}
\NormalTok{  knitr}\SpecialCharTok{::}\FunctionTok{asis\_output}\NormalTok{(}\StringTok{\textquotesingle{}}\SpecialCharTok{\textbackslash{}\textbackslash{}}\StringTok{url\{....\}\textquotesingle{}}\NormalTok{)}
\NormalTok{\} }\ControlFlowTok{else}\NormalTok{ \{}
\NormalTok{  knitr}\SpecialCharTok{::}\FunctionTok{include\_graphics}\NormalTok{(}\StringTok{"fig/Home\_01.gif"}\NormalTok{)}
\NormalTok{\}}
\end{Highlighting}
\end{Shaded}

\url{....}

\hypertarget{uxcf54uxb4dc-1}{%
\subsection{코드}\label{uxcf54uxb4dc-1}}

\begin{Shaded}
\begin{Highlighting}[]
\FunctionTok{move}\NormalTok{()}
\FunctionTok{move}\NormalTok{()}
\end{Highlighting}
\end{Shaded}

\hypertarget{home-02}{%
\section{Home 2}\label{home-02}}

\begin{itemize}
\tightlist
\item
  \href{https://reeborg.ca/reeborg.html?lang=ko-en\&mode=python\&menu=worlds\%2Fmenus\%2Freeborg_intro_en.json\&name=Home\%202\&url=worlds\%2Ftutorial_en\%2Fhome2.json}{문제
  바로가기}
\item
  리보그가 집을 떠나 있어 집으로 되돌아간다.

  \begin{itemize}
  \tightlist
  \item
    로봇의 좌표 (x, y) = (1, 3) 라서 로봇의 최종위치를 반드시 (x, y) =
    (1, 1) 으로 이동시킨다.
  \end{itemize}
\end{itemize}

\hypertarget{uxc2e4uxd589uxacb0uxacfc-2}{%
\subsection{실행결과}\label{uxc2e4uxd589uxacb0uxacfc-2}}

\begin{Shaded}
\begin{Highlighting}[]
\ControlFlowTok{if}\NormalTok{ (knitr}\SpecialCharTok{:::}\FunctionTok{is\_latex\_output}\NormalTok{()) \{}
\NormalTok{  knitr}\SpecialCharTok{::}\FunctionTok{asis\_output}\NormalTok{(}\StringTok{\textquotesingle{}}\SpecialCharTok{\textbackslash{}\textbackslash{}}\StringTok{url\{....\}\textquotesingle{}}\NormalTok{)}
\NormalTok{\} }\ControlFlowTok{else}\NormalTok{ \{}
\NormalTok{  knitr}\SpecialCharTok{::}\FunctionTok{include\_graphics}\NormalTok{(}\StringTok{"fig/Home\_02.gif"}\NormalTok{)}
\NormalTok{\}}
\end{Highlighting}
\end{Shaded}

\url{....}

\hypertarget{uxcf54uxb4dc-2}{%
\subsection{코드}\label{uxcf54uxb4dc-2}}

\begin{Shaded}
\begin{Highlighting}[]
\FunctionTok{move}\NormalTok{()}
\FunctionTok{move}\NormalTok{()}
\end{Highlighting}
\end{Shaded}

\hypertarget{home-03}{%
\section{Home 3}\label{home-03}}

\begin{itemize}
\tightlist
\item
  \href{https://reeborg.ca/reeborg.html?lang=ko-en\&mode=python\&menu=worlds\%2Fmenus\%2Freeborg_intro_en.json\&name=Home\%203\&url=worlds\%2Ftutorial_en\%2Fhome3.json}{문제
  바로가기}
\item
  리보그가 집을 떠나 있어 집으로 되돌아간다.

  \begin{itemize}
  \tightlist
  \item
    로봇의 좌표 (x, y) = (1, 3) 라서 로봇의 최종위치를 반드시 (x, y) =
    (2, 1) 으로 이동시킨다.
  \end{itemize}
\end{itemize}

\hypertarget{uxc2e4uxd589uxacb0uxacfc-3}{%
\subsection{실행결과}\label{uxc2e4uxd589uxacb0uxacfc-3}}

\begin{Shaded}
\begin{Highlighting}[]
\ControlFlowTok{if}\NormalTok{ (knitr}\SpecialCharTok{:::}\FunctionTok{is\_latex\_output}\NormalTok{()) \{}
\NormalTok{  knitr}\SpecialCharTok{::}\FunctionTok{asis\_output}\NormalTok{(}\StringTok{\textquotesingle{}}\SpecialCharTok{\textbackslash{}\textbackslash{}}\StringTok{url\{....\}\textquotesingle{}}\NormalTok{)}
\NormalTok{\} }\ControlFlowTok{else}\NormalTok{ \{}
\NormalTok{  knitr}\SpecialCharTok{::}\FunctionTok{include\_graphics}\NormalTok{(}\StringTok{"fig/Home\_03.gif"}\NormalTok{)}
\NormalTok{\}}
\end{Highlighting}
\end{Shaded}

\url{....}

\hypertarget{uxcf54uxb4dc-3}{%
\subsection{코드}\label{uxcf54uxb4dc-3}}

\begin{Shaded}
\begin{Highlighting}[]
\FunctionTok{move}\NormalTok{()}
\FunctionTok{move}\NormalTok{()}
\FunctionTok{turn\_left}\NormalTok{()}
\FunctionTok{move}\NormalTok{()}
\end{Highlighting}
\end{Shaded}

\hypertarget{home-04}{%
\section{Home 4}\label{home-04}}

\begin{itemize}
\tightlist
\item
  \href{https://reeborg.ca/reeborg.html?lang=ko-en\&mode=python\&menu=worlds\%2Fmenus\%2Freeborg_intro_en.json\&name=Home\%204\&url=worlds\%2Ftutorial_en\%2Fhome4.json}{문제
  바로가기}
\item
  리보그가 집을 떠나 있어 집으로 되돌아간다.

  \begin{itemize}
  \tightlist
  \item
    로봇의 좌표 (x, y) = (4, 1) 라서 로봇의 최종위치를 반드시 (x, y) =
    (5, 1) 으로 이동시킨다.
  \end{itemize}
\end{itemize}

\hypertarget{uxc2e4uxd589uxacb0uxacfc-4}{%
\subsection{실행결과}\label{uxc2e4uxd589uxacb0uxacfc-4}}

\begin{Shaded}
\begin{Highlighting}[]
\ControlFlowTok{if}\NormalTok{ (knitr}\SpecialCharTok{:::}\FunctionTok{is\_latex\_output}\NormalTok{()) \{}
\NormalTok{  knitr}\SpecialCharTok{::}\FunctionTok{asis\_output}\NormalTok{(}\StringTok{\textquotesingle{}}\SpecialCharTok{\textbackslash{}\textbackslash{}}\StringTok{url\{....\}\textquotesingle{}}\NormalTok{)}
\NormalTok{\} }\ControlFlowTok{else}\NormalTok{ \{}
\NormalTok{  knitr}\SpecialCharTok{::}\FunctionTok{include\_graphics}\NormalTok{(}\StringTok{"fig/Home\_04.gif"}\NormalTok{)}
\NormalTok{\}}
\end{Highlighting}
\end{Shaded}

\url{....}

\hypertarget{uxcf54uxb4dc-4}{%
\subsection{코드}\label{uxcf54uxb4dc-4}}

\begin{Shaded}
\begin{Highlighting}[]
\FunctionTok{move}\NormalTok{()}
\FunctionTok{move}\NormalTok{()}
\FunctionTok{move}\NormalTok{()}
\FunctionTok{turn\_left}\NormalTok{()}
\FunctionTok{move}\NormalTok{()}
\FunctionTok{move}\NormalTok{()}
\FunctionTok{move}\NormalTok{()}
\FunctionTok{turn\_left}\NormalTok{()}
\FunctionTok{turn\_left}\NormalTok{()}
\FunctionTok{turn\_left}\NormalTok{()}
\FunctionTok{move}\NormalTok{()}
\FunctionTok{turn\_left}\NormalTok{()}
\FunctionTok{turn\_left}\NormalTok{()}
\FunctionTok{turn\_left}\NormalTok{()}
\FunctionTok{move}\NormalTok{()}
\FunctionTok{move}\NormalTok{()}
\FunctionTok{move}\NormalTok{()}
\FunctionTok{turn\_left}\NormalTok{()}
\FunctionTok{move}\NormalTok{()}
\FunctionTok{move}\NormalTok{()}
\FunctionTok{move}\NormalTok{()}
\FunctionTok{turn\_left}\NormalTok{()}
\FunctionTok{turn\_left}\NormalTok{()}
\FunctionTok{turn\_left}\NormalTok{()}
\FunctionTok{move}\NormalTok{()}
\FunctionTok{turn\_left}\NormalTok{()}
\FunctionTok{turn\_left}\NormalTok{()}
\FunctionTok{turn\_left}\NormalTok{()}
\FunctionTok{move}\NormalTok{()}
\FunctionTok{move}\NormalTok{()}
\FunctionTok{move}\NormalTok{()}
\FunctionTok{turn\_left}\NormalTok{()}
\FunctionTok{move}\NormalTok{()}
\FunctionTok{move}\NormalTok{()}
\FunctionTok{move}\NormalTok{()}
\FunctionTok{turn\_left}\NormalTok{()}
\FunctionTok{turn\_left}\NormalTok{()}
\FunctionTok{turn\_left}\NormalTok{()}
\FunctionTok{move}\NormalTok{()}
\FunctionTok{turn\_left}\NormalTok{()}
\FunctionTok{turn\_left}\NormalTok{()}
\FunctionTok{turn\_left}\NormalTok{()}
\FunctionTok{move}\NormalTok{()}
\FunctionTok{move}\NormalTok{()}
\FunctionTok{move}\NormalTok{()}
\FunctionTok{turn\_left}\NormalTok{()}
\FunctionTok{move}\NormalTok{()}
\FunctionTok{move}\NormalTok{()}
\FunctionTok{move}\NormalTok{()}
\end{Highlighting}
\end{Shaded}

\hypertarget{around}{%
\chapter{돌아다니기 (Around)}\label{around}}

\hypertarget{around-01}{%
\section{Around 1}\label{around-01}}

\begin{itemize}
\tightlist
\item
  \href{https://reeborg.ca/reeborg.html?lang=ko-en\&mode=python\&menu=worlds\%2Fmenus\%2Freeborg_intro_en.json\&name=Around\%201\&url=worlds\%2Ftutorial_en\%2Faround1.json}{문제
  바로가기}
\item
  리보그를 세상한바퀴 돌게 만드는 것이다.

  \begin{itemize}
  \tightlist
  \item
    로봇의 좌표 (x, y) = (1, 1) 라서 로봇의 최종위치를 한바퀴 돌아
    반드시 (x, y) = (1, 1) 으로 이동시킨다.
  \end{itemize}
\item
  난이도: 1

  \begin{itemize}
  \tightlist
  \item
    함수: \texttt{move()}, \texttt{turn\_left()}, \texttt{put()}
  \end{itemize}
\item
  난이도: 2

  \begin{itemize}
  \tightlist
  \item
    함수: \texttt{move()}, \texttt{turn\_left()}, \texttt{put()}
  \item
    테스트 함수: \texttt{front\_is\_clear()},
    \texttt{wall\_in\_front()}, \texttt{object\_here()}
  \item
    반복 \texttt{while}, \texttt{repeat}/\texttt{for} 루프와 제어
    \texttt{if}문
  \end{itemize}
\end{itemize}

\hypertarget{uxc2e4uxd589uxacb0uxacfc-5}{%
\subsection{실행결과}\label{uxc2e4uxd589uxacb0uxacfc-5}}

\begin{Shaded}
\begin{Highlighting}[]
\ControlFlowTok{if}\NormalTok{ (knitr}\SpecialCharTok{:::}\FunctionTok{is\_latex\_output}\NormalTok{()) \{}
\NormalTok{  knitr}\SpecialCharTok{::}\FunctionTok{asis\_output}\NormalTok{(}\StringTok{\textquotesingle{}}\SpecialCharTok{\textbackslash{}\textbackslash{}}\StringTok{url\{....\}\textquotesingle{}}\NormalTok{)}
\NormalTok{\} }\ControlFlowTok{else}\NormalTok{ \{}
\NormalTok{  knitr}\SpecialCharTok{::}\FunctionTok{include\_graphics}\NormalTok{(}\StringTok{"fig/Around\_01.gif"}\NormalTok{)}
\NormalTok{\}}
\end{Highlighting}
\end{Shaded}

\url{....}

\hypertarget{uxcf54uxb4dc-5}{%
\subsection{코드}\label{uxcf54uxb4dc-5}}

\begin{Shaded}
\begin{Highlighting}[]
\NormalTok{repeat }\DecValTok{4}\NormalTok{ :}
\NormalTok{    repeat }\DecValTok{9}\NormalTok{ :}
\NormalTok{        move()}
\NormalTok{    turn\_left()}
\end{Highlighting}
\end{Shaded}

\hypertarget{around-02}{%
\section{Around 2}\label{around-02}}

리보그가 토큰(token)을 가지고 다니는 점을 이용하여 리보그 세상을 한바퀴
삥 둘러 돌아다닌 후에 제자리로 돌아온 위치를 표식하고 이를 프로그램 종료
조건으로 설정한다.

\begin{itemize}
\tightlist
\item
  \href{https://reeborg.ca/reeborg.html?lang=ko-en\&mode=python\&menu=worlds\%2Fmenus\%2Freeborg_intro_en.json\&name=Around\%202\&url=worlds\%2Ftutorial_en\%2Faround2.json}{문제
  바로가기}
\item
  리보그를 세상한바퀴 돌게 만드는 것이다.

  \begin{itemize}
  \tightlist
  \item
    로봇의 좌표 (x, y) 가 임의 랜덤으로 주어졌기 때문에 로봇의
    최종위치를 한바퀴 돌아 반드시 (x, y) 랜덤 위치로 이동시킨다.
  \end{itemize}
\item
  난이도: 3

  \begin{itemize}
  \tightlist
  \item
    함수: \texttt{move()}, \texttt{turn\_left()}, \texttt{put()}
  \item
    테스트 함수: \texttt{front\_is\_clear()},
    \texttt{wall\_in\_front()}, \texttt{right\_is\_clear()},
    \texttt{wall\_on\_right()}, \texttt{object\_here()}
  \item
    반복 \texttt{while}, \texttt{repeat}/\texttt{for} 루프와 제어
    \texttt{if}문
  \end{itemize}
\end{itemize}

\hypertarget{uxc2e4uxd589uxacb0uxacfc-6}{%
\subsection{실행결과}\label{uxc2e4uxd589uxacb0uxacfc-6}}

\begin{Shaded}
\begin{Highlighting}[]
\ControlFlowTok{if}\NormalTok{ (knitr}\SpecialCharTok{:::}\FunctionTok{is\_latex\_output}\NormalTok{()) \{}
\NormalTok{  knitr}\SpecialCharTok{::}\FunctionTok{asis\_output}\NormalTok{(}\StringTok{\textquotesingle{}}\SpecialCharTok{\textbackslash{}\textbackslash{}}\StringTok{url\{....\}\textquotesingle{}}\NormalTok{)}
\NormalTok{\} }\ControlFlowTok{else}\NormalTok{ \{}
\NormalTok{  knitr}\SpecialCharTok{::}\FunctionTok{include\_graphics}\NormalTok{(}\StringTok{"fig/Around\_02.gif"}\NormalTok{)}
\NormalTok{\}}
\end{Highlighting}
\end{Shaded}

\url{....}

\hypertarget{uxcf54uxb4dc-6}{%
\subsection{코드}\label{uxcf54uxb4dc-6}}

\begin{Shaded}
\begin{Highlighting}[]
\KeywordTok{def}\NormalTok{ turn\_right():}
\NormalTok{    turn\_left()}
\NormalTok{    turn\_left()}
\NormalTok{    turn\_left()}

\CommentTok{\# 목표지점 표식    }
\KeywordTok{def}\NormalTok{ drop\_token():}
\NormalTok{    put(}\StringTok{"token"}\NormalTok{)}

\NormalTok{drop\_token()}
\NormalTok{move()}

\ControlFlowTok{while} \KeywordTok{not}\NormalTok{ object\_here():}
\NormalTok{    move()}
    \ControlFlowTok{if}\NormalTok{ wall\_in\_front() :}
\NormalTok{        turn\_left()}
    \ControlFlowTok{if}\NormalTok{ right\_is\_clear() :}
\NormalTok{        turn\_right()}
\NormalTok{        move()}
\end{Highlighting}
\end{Shaded}

\hypertarget{around-03}{%
\section{Around 3}\label{around-03}}

\protect\hyperlink{around-02}{Around 2}와 마찬가지로 토큰을 떨어뜨려
표식을 하고 자리를 한칸 이동한 후에 돌아다니는 작업을 수행한다.

\begin{itemize}
\tightlist
\item
  \href{https://reeborg.ca/reeborg.html?lang=ko-en\&mode=python\&menu=worlds\%2Fmenus\%2Freeborg_intro_en.json\&name=Around\%203\&url=worlds\%2Ftutorial_en\%2Faround3.json}{문제
  바로가기}
\item
  리보그를 세상한바퀴 돌게 만드는 것이다.

  \begin{itemize}
  \tightlist
  \item
    로봇의 좌표 (x, y) 가 임의 랜덤으로 주어졌기 때문에 로봇의
    최종위치를 한바퀴 돌아 반드시 (x, y) 랜덤 위치로 이동시킨다.
  \end{itemize}
\item
  난이도: 4

  \begin{itemize}
  \tightlist
  \item
    함수: \texttt{move()}, \texttt{turn\_left()}, \texttt{put()}
  \item
    테스트 함수: \texttt{front\_is\_clear()},
    \texttt{wall\_in\_front()}, \texttt{right\_is\_clear()},
    \texttt{wall\_on\_right()}, \texttt{object\_here()}
  \item
    반복 \texttt{while}, \texttt{repeat}/\texttt{for} 루프와 제어
    \texttt{if}문
  \end{itemize}
\end{itemize}

\hypertarget{uxc2e4uxd589uxacb0uxacfc-7}{%
\subsection{실행결과}\label{uxc2e4uxd589uxacb0uxacfc-7}}

\begin{Shaded}
\begin{Highlighting}[]
\ControlFlowTok{if}\NormalTok{ (knitr}\SpecialCharTok{:::}\FunctionTok{is\_latex\_output}\NormalTok{()) \{}
\NormalTok{  knitr}\SpecialCharTok{::}\FunctionTok{asis\_output}\NormalTok{(}\StringTok{\textquotesingle{}}\SpecialCharTok{\textbackslash{}\textbackslash{}}\StringTok{url\{....\}\textquotesingle{}}\NormalTok{)}
\NormalTok{\} }\ControlFlowTok{else}\NormalTok{ \{}
\NormalTok{  knitr}\SpecialCharTok{::}\FunctionTok{include\_graphics}\NormalTok{(}\StringTok{"fig/Around\_03.gif"}\NormalTok{)}
\NormalTok{\}}
\end{Highlighting}
\end{Shaded}

\url{....}

\hypertarget{uxcf54uxb4dc-7}{%
\subsection{코드}\label{uxcf54uxb4dc-7}}

\begin{Shaded}
\begin{Highlighting}[]
\KeywordTok{def}\NormalTok{ turn\_right():}
\NormalTok{    turn\_left()}
\NormalTok{    turn\_left()}
\NormalTok{    turn\_left()}

\CommentTok{\# 목표지점 표식    }
\KeywordTok{def}\NormalTok{ drop\_token():}
\NormalTok{    put(}\StringTok{"token"}\NormalTok{)}

\NormalTok{drop\_token()}
\NormalTok{turn\_left()}
\NormalTok{move()}

\ControlFlowTok{while} \KeywordTok{not}\NormalTok{ object\_here():}
    \ControlFlowTok{if}\NormalTok{ right\_is\_clear():}
\NormalTok{        turn\_right()}
\NormalTok{        move()}
    \ControlFlowTok{elif}\NormalTok{ front\_is\_clear():}
\NormalTok{        move()}
    \ControlFlowTok{else}\NormalTok{:}
\NormalTok{        turn\_left()}
\end{Highlighting}
\end{Shaded}

\hypertarget{around-04}{%
\section{Around 4}\label{around-04}}

본 문제를 해결하게 되면 앞선 Around 1,2,3번 문제도 작성한 프로그램으로
일반화시켜 해결할 수 있다.

\begin{itemize}
\tightlist
\item
  \href{https://reeborg.ca/reeborg.html?lang=ko-en\&mode=python\&menu=worlds\%2Fmenus\%2Freeborg_intro_en.json\&name=Around\%203\&url=worlds\%2Ftutorial_en\%2Faround3.json}{문제
  바로가기}
\item
  리보그를 세상한바퀴 돌게 만드는 것이다.

  \begin{itemize}
  \tightlist
  \item
    로봇의 좌표 (x, y) 가 임의 랜덤으로 주어졌기 때문에 로봇의
    최종위치를 한바퀴 돌아 반드시 (x, y) 랜덤 위치로 이동시킨다.
  \end{itemize}
\item
  난이도: 5

  \begin{itemize}
  \tightlist
  \item
    함수: \texttt{move()}, \texttt{turn\_left()}, \texttt{put()}
  \item
    테스트 함수: \texttt{front\_is\_clear()},
    \texttt{wall\_in\_front()}, \texttt{right\_is\_clear()},
    \texttt{wall\_on\_right()}, \texttt{object\_here()}
  \item
    반복 \texttt{while}, \texttt{if/elif/else} 문
  \end{itemize}
\end{itemize}

\hypertarget{uxc2e4uxd589uxacb0uxacfc-8}{%
\subsection{실행결과}\label{uxc2e4uxd589uxacb0uxacfc-8}}

\begin{Shaded}
\begin{Highlighting}[]
\ControlFlowTok{if}\NormalTok{ (knitr}\SpecialCharTok{:::}\FunctionTok{is\_latex\_output}\NormalTok{()) \{}
\NormalTok{  knitr}\SpecialCharTok{::}\FunctionTok{asis\_output}\NormalTok{(}\StringTok{\textquotesingle{}}\SpecialCharTok{\textbackslash{}\textbackslash{}}\StringTok{url\{....\}\textquotesingle{}}\NormalTok{)}
\NormalTok{\} }\ControlFlowTok{else}\NormalTok{ \{}
\NormalTok{  knitr}\SpecialCharTok{::}\FunctionTok{include\_graphics}\NormalTok{(}\StringTok{"fig/Around\_04.gif"}\NormalTok{)}
\NormalTok{\}}
\end{Highlighting}
\end{Shaded}

\url{....}

\hypertarget{uxcf54uxb4dc-8}{%
\subsection{코드}\label{uxcf54uxb4dc-8}}

\begin{Shaded}
\begin{Highlighting}[]
\KeywordTok{def}\NormalTok{ turn\_right():}
\NormalTok{    turn\_left()}
\NormalTok{    turn\_left()}
\NormalTok{    turn\_left()}

\KeywordTok{def}\NormalTok{ turn\_around():}
\NormalTok{    turn\_left()}
\NormalTok{    turn\_left()}
    
    
\CommentTok{\# 목표지점 표식    }
\KeywordTok{def}\NormalTok{ drop\_token():}
\NormalTok{    put(}\StringTok{"token"}\NormalTok{)}

\NormalTok{drop\_token()}
\NormalTok{turn\_around()}
\NormalTok{move()}

\ControlFlowTok{while} \KeywordTok{not}\NormalTok{ object\_here():}
    \ControlFlowTok{if}\NormalTok{ right\_is\_clear():}
\NormalTok{        turn\_right()}
\NormalTok{        move()}
    \ControlFlowTok{elif}\NormalTok{ front\_is\_clear():}
\NormalTok{        move()}
    \ControlFlowTok{else}\NormalTok{:}
\NormalTok{        turn\_left()}
\end{Highlighting}
\end{Shaded}

\hypertarget{newspaper}{%
\chapter{신문배달}\label{newspaper}}

\hypertarget{newspaper-01}{%
\section{신문배달 1}\label{newspaper-01}}

\begin{itemize}
\tightlist
\item
  \href{https://reeborg.ca/reeborg.html?lang=ko-en\&mode=python\&menu=worlds\%2Fmenus\%2Freeborg_intro_en.json\&name=Newspaper\%200\&url=worlds\%2Ftutorial_en\%2Fnewspaper0.json}{문제
  바로가기}
\item
  선행 지식

  \begin{itemize}
  \tightlist
  \item
    기본 함수 : \texttt{move()}, \texttt{turn\_left()}, \texttt{take()},
    \texttt{put()}
  \end{itemize}
\item
  난이도: 1
\end{itemize}

\hypertarget{uxc2e4uxd589uxacb0uxacfc-9}{%
\subsection{실행결과}\label{uxc2e4uxd589uxacb0uxacfc-9}}

\begin{Shaded}
\begin{Highlighting}[]
\ControlFlowTok{if}\NormalTok{ (knitr}\SpecialCharTok{:::}\FunctionTok{is\_latex\_output}\NormalTok{()) \{}
\NormalTok{  knitr}\SpecialCharTok{::}\FunctionTok{asis\_output}\NormalTok{(}\StringTok{\textquotesingle{}}\SpecialCharTok{\textbackslash{}\textbackslash{}}\StringTok{url\{....\}\textquotesingle{}}\NormalTok{)}
\NormalTok{\} }\ControlFlowTok{else}\NormalTok{ \{}
\NormalTok{  knitr}\SpecialCharTok{::}\FunctionTok{include\_graphics}\NormalTok{(}\StringTok{"fig/Newspaper\_01.gif"}\NormalTok{)}
\NormalTok{\}}
\end{Highlighting}
\end{Shaded}

\url{....}

\hypertarget{uxcf54uxb4dc-9}{%
\subsection{코드}\label{uxcf54uxb4dc-9}}

\begin{Shaded}
\begin{Highlighting}[]
\KeywordTok{def}\NormalTok{ turn\_right():}
\NormalTok{    turn\_left()}
\NormalTok{    turn\_left()}
\NormalTok{    turn\_left()}

\CommentTok{\# 계단 올라가기 {-}{-}{-}{-}    }
\NormalTok{take()}
\NormalTok{turn\_left()}
\NormalTok{move()}
\NormalTok{turn\_right()}

\NormalTok{move()}
\NormalTok{move()}

\NormalTok{turn\_left()}
\NormalTok{move()}
\NormalTok{turn\_right()}

\NormalTok{move()}
\NormalTok{move()}

\NormalTok{turn\_left()}
\NormalTok{move()}
\NormalTok{turn\_right()}

\NormalTok{move()}
\NormalTok{move()}
\NormalTok{put()}

\CommentTok{\# 계단 내려오기 {-}{-}{-}{-}}
\NormalTok{turn\_left()}
\NormalTok{turn\_left()}
\NormalTok{move()}
\NormalTok{move()}
\NormalTok{turn\_left()}
\NormalTok{move()}

\NormalTok{turn\_right()}
\NormalTok{move()}
\NormalTok{move()}

\NormalTok{turn\_left()}
\NormalTok{move()}
\NormalTok{turn\_right()}

\NormalTok{move()}
\NormalTok{move()}

\NormalTok{turn\_left()}
\NormalTok{move()}
\end{Highlighting}
\end{Shaded}

\hypertarget{newspaper-02}{%
\section{신문배달 2}\label{newspaper-02}}

\begin{itemize}
\tightlist
\item
  \href{https://reeborg.ca/reeborg.html?lang=ko-en\&mode=python\&menu=worlds\%2Fmenus\%2Freeborg_intro_en.json\&name=Newspaper\%200\&url=worlds\%2Ftutorial_en\%2Fnewspaper0.json}{문제
  바로가기}
\item
  선행 지식

  \begin{itemize}
  \tightlist
  \item
    기본 함수 : \texttt{move()}, \texttt{turn\_left()}, \texttt{take()},
    \texttt{put()}
  \end{itemize}
\item
  난이도: 3
\item
  힌트: \texttt{up\_three\_steps()}, \texttt{down\_three\_steps()},
  \texttt{turn\_around()} 함수를 제작하여 MS. Ada Lovelace에게 전달되는
  신문업무를 모듈화시키고 \texttt{take("token")} 함수를 호출하여 댓가를
  선물로 받아 집으로 돌아온다.
\end{itemize}

\hypertarget{uxc2e4uxd589uxacb0uxacfc-10}{%
\subsection{실행결과}\label{uxc2e4uxd589uxacb0uxacfc-10}}

\begin{Shaded}
\begin{Highlighting}[]
\ControlFlowTok{if}\NormalTok{ (knitr}\SpecialCharTok{:::}\FunctionTok{is\_latex\_output}\NormalTok{()) \{}
\NormalTok{  knitr}\SpecialCharTok{::}\FunctionTok{asis\_output}\NormalTok{(}\StringTok{\textquotesingle{}}\SpecialCharTok{\textbackslash{}\textbackslash{}}\StringTok{url\{....\}\textquotesingle{}}\NormalTok{)}
\NormalTok{\} }\ControlFlowTok{else}\NormalTok{ \{}
\NormalTok{  knitr}\SpecialCharTok{::}\FunctionTok{include\_graphics}\NormalTok{(}\StringTok{"fig/Newspaper\_02.gif"}\NormalTok{)}
\NormalTok{\}}
\end{Highlighting}
\end{Shaded}

\url{....}

\hypertarget{uxcf54uxb4dc-10}{%
\subsection{코드}\label{uxcf54uxb4dc-10}}

\begin{Shaded}
\begin{Highlighting}[]
\KeywordTok{def}\NormalTok{ turn\_right():}
\NormalTok{    turn\_left()}
\NormalTok{    turn\_left()}
\NormalTok{    turn\_left()}

\KeywordTok{def}\NormalTok{ turn\_around():}
\NormalTok{    turn\_left()}
\NormalTok{    turn\_left()}

\KeywordTok{def}\NormalTok{ up\_three\_steps():}
\NormalTok{    repeat }\DecValTok{3}\NormalTok{:}
\NormalTok{        turn\_left()}
\NormalTok{        move()}
\NormalTok{        turn\_right()}
\NormalTok{        move()}
\NormalTok{        move()}

\KeywordTok{def}\NormalTok{ down\_three\_steps():}
\NormalTok{    repeat }\DecValTok{3}\NormalTok{:}
\NormalTok{        move()}
\NormalTok{        move()}
\NormalTok{        turn\_left()}
\NormalTok{        move()}
\NormalTok{        turn\_right()}

\CommentTok{\# 신문 배달 시작}
\NormalTok{take()}
\CommentTok{\# 계단 올라가서 러브레이스 도착}
\NormalTok{up\_three\_steps()}
\CommentTok{\# 신문 높고 토큰 받기}
\NormalTok{put()}
\ControlFlowTok{while}\NormalTok{ object\_here(}\StringTok{"token"}\NormalTok{):}
\NormalTok{    take(}\StringTok{"token"}\NormalTok{)}
\CommentTok{\# 되돌아 집에 오기    }
\NormalTok{turn\_around()}
\NormalTok{down\_three\_steps()}
\end{Highlighting}
\end{Shaded}

\hypertarget{hurdle}{%
\chapter{장애물 넘기}\label{hurdle}}

\hypertarget{hurdle-01}{%
\section{장애물 1}\label{hurdle-01}}

\begin{itemize}
\tightlist
\item
  \href{https://reeborg.ca/reeborg.html?lang=ko-en\&mode=python\&menu=worlds\%2Fmenus\%2Freeborg_intro_en.json\&name=Hurdle\%201\&url=worlds\%2Ftutorial_en\%2Fhurdle1.json}{문제
  바로가기}
\item
  선행 지식

  \begin{itemize}
  \tightlist
  \item
    기본 함수 : \texttt{move()}, \texttt{turn\_left()}
  \end{itemize}
\item
  힌트: \texttt{jump()} 함수를 작성해서 프로그램 코드를 단순화시킬 수
  있다.
\item
  난이도: 2
\item
  참고: \href{https://reeborg.ca/docs/ko/variables/harvest3.html}{리보그
  세상 - 추가된 추수 도전과제}
\end{itemize}

\hypertarget{uxc2e4uxd589uxacb0uxacfc-11}{%
\subsection{실행결과}\label{uxc2e4uxd589uxacb0uxacfc-11}}

\begin{Shaded}
\begin{Highlighting}[]
\ControlFlowTok{if}\NormalTok{ (knitr}\SpecialCharTok{:::}\FunctionTok{is\_latex\_output}\NormalTok{()) \{}
\NormalTok{  knitr}\SpecialCharTok{::}\FunctionTok{asis\_output}\NormalTok{(}\StringTok{\textquotesingle{}}\SpecialCharTok{\textbackslash{}\textbackslash{}}\StringTok{url\{....\}\textquotesingle{}}\NormalTok{)}
\NormalTok{\} }\ControlFlowTok{else}\NormalTok{ \{}
\NormalTok{  knitr}\SpecialCharTok{::}\FunctionTok{include\_graphics}\NormalTok{(}\StringTok{"fig/Hurdle\_01.gif"}\NormalTok{)}
\NormalTok{\}}
\end{Highlighting}
\end{Shaded}

\url{....}

\hypertarget{uxcf54uxb4dc-11}{%
\subsection{코드}\label{uxcf54uxb4dc-11}}

\begin{Shaded}
\begin{Highlighting}[]
\NormalTok{def }\FunctionTok{turn\_right}\NormalTok{()}\SpecialCharTok{:}
    \FunctionTok{turn\_left}\NormalTok{()}
    \FunctionTok{turn\_left}\NormalTok{()}
    \FunctionTok{turn\_left}\NormalTok{()}

\NormalTok{def }\FunctionTok{jump\_over\_hurdle}\NormalTok{()}\SpecialCharTok{:}    
    \FunctionTok{move}\NormalTok{()}
    \FunctionTok{turn\_left}\NormalTok{()}
    \FunctionTok{move}\NormalTok{()}
    \FunctionTok{turn\_right}\NormalTok{()}
    \FunctionTok{move}\NormalTok{()}
    \FunctionTok{turn\_right}\NormalTok{()}
    \FunctionTok{move}\NormalTok{()}
    \FunctionTok{turn\_left}\NormalTok{()}

\CommentTok{\# jump\_over\_hurdle()    }

\ControlFlowTok{repeat} \DecValTok{6}\SpecialCharTok{:}
   \FunctionTok{jump\_over\_hurdle}\NormalTok{()}
\end{Highlighting}
\end{Shaded}

\hypertarget{hurdle-02}{%
\section{장애물 2}\label{hurdle-02}}

\begin{itemize}
\tightlist
\item
  \href{https://reeborg.ca/reeborg.html?lang=ko-en\&mode=python\&menu=worlds\%2Fmenus\%2Freeborg_intro_en.json\&name=Hurdle\%202\&url=worlds\%2Ftutorial_en\%2Fhurdle2.json}{문제
  바로가기}
\item
  선행 지식

  \begin{itemize}
  \tightlist
  \item
    기본 함수 : \texttt{move()}, \texttt{turn\_left()}
  \item
    조건 / 테스트 : \texttt{at\_goal()} 혹은 부정(negation)
  \item
    반복: \texttt{while()}
  \end{itemize}
\item
  난이도: 3
\end{itemize}

\hypertarget{uxc2e4uxd589uxacb0uxacfc-12}{%
\subsection{실행결과}\label{uxc2e4uxd589uxacb0uxacfc-12}}

\begin{Shaded}
\begin{Highlighting}[]
\ControlFlowTok{if}\NormalTok{ (knitr}\SpecialCharTok{:::}\FunctionTok{is\_latex\_output}\NormalTok{()) \{}
\NormalTok{  knitr}\SpecialCharTok{::}\FunctionTok{asis\_output}\NormalTok{(}\StringTok{\textquotesingle{}}\SpecialCharTok{\textbackslash{}\textbackslash{}}\StringTok{url\{....\}\textquotesingle{}}\NormalTok{)}
\NormalTok{\} }\ControlFlowTok{else}\NormalTok{ \{}
\NormalTok{  knitr}\SpecialCharTok{::}\FunctionTok{include\_graphics}\NormalTok{(}\StringTok{"fig/Hurdle\_02.gif"}\NormalTok{)}
\NormalTok{\}}
\end{Highlighting}
\end{Shaded}

\url{....}

\hypertarget{uxcf54uxb4dc-12}{%
\subsection{코드}\label{uxcf54uxb4dc-12}}

\begin{Shaded}
\begin{Highlighting}[]
\NormalTok{def }\FunctionTok{turn\_right}\NormalTok{()}\SpecialCharTok{:}
    \FunctionTok{turn\_left}\NormalTok{()}
    \FunctionTok{turn\_left}\NormalTok{()}
    \FunctionTok{turn\_left}\NormalTok{()}

\NormalTok{def }\FunctionTok{jump\_over\_hurdle}\NormalTok{()}\SpecialCharTok{:}    
    \FunctionTok{move}\NormalTok{()}
    \FunctionTok{turn\_left}\NormalTok{()}
    \FunctionTok{move}\NormalTok{()}
    \FunctionTok{turn\_right}\NormalTok{()}
    \FunctionTok{move}\NormalTok{()}
    \FunctionTok{turn\_right}\NormalTok{()}
    \FunctionTok{move}\NormalTok{()}
    \FunctionTok{turn\_left}\NormalTok{()}

\CommentTok{\# jump\_over\_hurdle()    }

\ControlFlowTok{while}\NormalTok{ not }\FunctionTok{at\_goal}\NormalTok{()}\SpecialCharTok{:}
    \FunctionTok{jump\_over\_hurdle}\NormalTok{()}
\end{Highlighting}
\end{Shaded}

\hypertarget{hurdle-03}{%
\section{장애물 3}\label{hurdle-03}}

\begin{itemize}
\tightlist
\item
  \href{https://reeborg.ca/reeborg.html?lang=ko-en\&mode=python\&menu=worlds\%2Fmenus\%2Freeborg_intro_en.json\&name=Hurdle\%203\&url=worlds\%2Ftutorial_en\%2Fhurdle3.json}{문제
  바로가기}
\item
  선행 지식

  \begin{itemize}
  \tightlist
  \item
    기본 함수 : \texttt{move()}, \texttt{turn\_left()}
  \item
    조건 / 테스트 : \texttt{at\_goal()}, \texttt{front\_is\_clear()},
    \texttt{wall\_in\_front()}, 혹은 부정(negation)
  \item
    반복과 제어: \texttt{while()} 루프와 \texttt{if} 문
  \end{itemize}
\item
  난이도: 4
\end{itemize}

\hypertarget{uxc2e4uxd589uxacb0uxacfc-13}{%
\subsection{실행결과}\label{uxc2e4uxd589uxacb0uxacfc-13}}

\begin{Shaded}
\begin{Highlighting}[]
\ControlFlowTok{if}\NormalTok{ (knitr}\SpecialCharTok{:::}\FunctionTok{is\_latex\_output}\NormalTok{()) \{}
\NormalTok{  knitr}\SpecialCharTok{::}\FunctionTok{asis\_output}\NormalTok{(}\StringTok{\textquotesingle{}}\SpecialCharTok{\textbackslash{}\textbackslash{}}\StringTok{url\{....\}\textquotesingle{}}\NormalTok{)}
\NormalTok{\} }\ControlFlowTok{else}\NormalTok{ \{}
\NormalTok{  knitr}\SpecialCharTok{::}\FunctionTok{include\_graphics}\NormalTok{(}\StringTok{"fig/Hurdle\_03.gif"}\NormalTok{)}
\NormalTok{\}}
\end{Highlighting}
\end{Shaded}

\url{....}

\hypertarget{uxcf54uxb4dc-13}{%
\subsection{코드}\label{uxcf54uxb4dc-13}}

\begin{Shaded}
\begin{Highlighting}[]
\NormalTok{def }\FunctionTok{turn\_right}\NormalTok{()}\SpecialCharTok{:}
    \FunctionTok{turn\_left}\NormalTok{()}
    \FunctionTok{turn\_left}\NormalTok{()}
    \FunctionTok{turn\_left}\NormalTok{()}

\NormalTok{def }\FunctionTok{jump\_over\_hurdle}\NormalTok{()}\SpecialCharTok{:}
    \CommentTok{\# move() \textless{}{-}{-} 일반화를 위해 제거}
    \FunctionTok{turn\_left}\NormalTok{()}
    \FunctionTok{move}\NormalTok{()}
    \FunctionTok{turn\_right}\NormalTok{()}
    \FunctionTok{move}\NormalTok{()}
    \FunctionTok{turn\_right}\NormalTok{()}
    \FunctionTok{move}\NormalTok{()}
    \FunctionTok{turn\_left}\NormalTok{()}

\CommentTok{\# jump\_over\_hurdle()    }

\ControlFlowTok{while}\NormalTok{ not }\FunctionTok{at\_goal}\NormalTok{()}\SpecialCharTok{:}
    \ControlFlowTok{if} \FunctionTok{front\_is\_clear}\NormalTok{()}\SpecialCharTok{:}
        \FunctionTok{move}\NormalTok{()}
\NormalTok{    elif }\FunctionTok{wall\_in\_front}\NormalTok{()}\SpecialCharTok{:}
        \FunctionTok{jump\_over\_hurdle}\NormalTok{()}
\end{Highlighting}
\end{Shaded}

\hypertarget{hurdle-04}{%
\section{장애물 4}\label{hurdle-04}}

\begin{itemize}
\tightlist
\item
  \href{https://reeborg.ca/reeborg.html?lang=ko-en\&mode=python\&menu=worlds\%2Fmenus\%2Freeborg_intro_en.json\&name=Hurdle\%204\&url=worlds\%2Ftutorial_en\%2Fhurdle4.json}{문제
  바로가기}
\item
  선행 지식

  \begin{itemize}
  \tightlist
  \item
    기본 함수 : \texttt{move()}, \texttt{turn\_left()}
  \item
    조건 / 테스트 : \texttt{at\_goal()}, \texttt{front\_is\_clear()},
    \texttt{wall\_in\_front()}, 혹은 부정(negation)
  \item
    반복과 제어: \texttt{while()} 루프와 \texttt{if} 문
  \end{itemize}
\item
  난이도: 4.5
\item
  장애물 4 프로그램은 장애물 1, 2, 3 프로그램도 정상 동작시킬 수 있어야
  한다.
\end{itemize}

\hypertarget{uxc2e4uxd589uxacb0uxacfc-14}{%
\subsection{실행결과}\label{uxc2e4uxd589uxacb0uxacfc-14}}

\begin{Shaded}
\begin{Highlighting}[]
\ControlFlowTok{if}\NormalTok{ (knitr}\SpecialCharTok{:::}\FunctionTok{is\_latex\_output}\NormalTok{()) \{}
\NormalTok{  knitr}\SpecialCharTok{::}\FunctionTok{asis\_output}\NormalTok{(}\StringTok{\textquotesingle{}}\SpecialCharTok{\textbackslash{}\textbackslash{}}\StringTok{url\{....\}\textquotesingle{}}\NormalTok{)}
\NormalTok{\} }\ControlFlowTok{else}\NormalTok{ \{}
\NormalTok{  knitr}\SpecialCharTok{::}\FunctionTok{include\_graphics}\NormalTok{(}\StringTok{"fig/Hurdle\_04.gif"}\NormalTok{)}
\NormalTok{\}}
\end{Highlighting}
\end{Shaded}

\url{....}

\hypertarget{uxcf54uxb4dc-14}{%
\subsection{코드}\label{uxcf54uxb4dc-14}}

\begin{Shaded}
\begin{Highlighting}[]
\NormalTok{def }\FunctionTok{turn\_right}\NormalTok{()}\SpecialCharTok{:}
    \FunctionTok{turn\_left}\NormalTok{()}
    \FunctionTok{turn\_left}\NormalTok{()}
    \FunctionTok{turn\_left}\NormalTok{()}

\NormalTok{def }\FunctionTok{jump\_over\_hurdles}\NormalTok{()}\SpecialCharTok{:}
    \CommentTok{\# 장애물 위쪽 올라가기}
    \ControlFlowTok{if} \FunctionTok{wall\_in\_front}\NormalTok{()}\SpecialCharTok{:}
        \FunctionTok{turn\_left}\NormalTok{()}
        \ControlFlowTok{while}\NormalTok{ not }\FunctionTok{right\_is\_clear}\NormalTok{()}\SpecialCharTok{:}
            \FunctionTok{move}\NormalTok{()}
    \CommentTok{\# 장애물 위를 넘어가기            }
    \FunctionTok{turn\_right}\NormalTok{()}
    \FunctionTok{move}\NormalTok{()}
    \FunctionTok{turn\_right}\NormalTok{()}
    \CommentTok{\# 장애물 내려오기}
    \ControlFlowTok{while} \FunctionTok{front\_is\_clear}\NormalTok{()}\SpecialCharTok{:}
        \FunctionTok{move}\NormalTok{()}
    \CommentTok{\# 다시 경주자세로 자세 갖추기}
    \FunctionTok{turn\_left}\NormalTok{()}

\ControlFlowTok{while}\NormalTok{ not }\FunctionTok{at\_goal}\NormalTok{()}\SpecialCharTok{:}
    \ControlFlowTok{if} \FunctionTok{front\_is\_clear}\NormalTok{()}\SpecialCharTok{:}
        \FunctionTok{move}\NormalTok{()}
\NormalTok{    elif }\FunctionTok{wall\_in\_front}\NormalTok{()}\SpecialCharTok{:}
        \FunctionTok{jump\_over\_hurdles}\NormalTok{()}
\end{Highlighting}
\end{Shaded}

\hypertarget{rain}{%
\chapter{토큰}\label{rain}}

\hypertarget{token-01}{%
\section{토큰 이동 1}\label{token-01}}

\begin{itemize}
\tightlist
\item
  \href{https://reeborg.ca/reeborg.html?lang=ko-en\&mode=python\&menu=worlds\%2Fmenus\%2Freeborg_intro_en.json\&name=Tokens\%201\&url=worlds\%2Ftutorial_en\%2Ftokens1.json}{문제
  바로가기}
\item
  선행 지식

  \begin{itemize}
  \tightlist
  \item
    기본 함수 : \texttt{move()}, \texttt{put()}, \texttt{take()}
  \end{itemize}
\item
  난이도: 1
\end{itemize}

\hypertarget{uxc2e4uxd589uxacb0uxacfc-15}{%
\subsection{실행결과}\label{uxc2e4uxd589uxacb0uxacfc-15}}

\begin{Shaded}
\begin{Highlighting}[]
\ControlFlowTok{if}\NormalTok{ (knitr}\SpecialCharTok{:::}\FunctionTok{is\_latex\_output}\NormalTok{()) \{}
\NormalTok{  knitr}\SpecialCharTok{::}\FunctionTok{asis\_output}\NormalTok{(}\StringTok{\textquotesingle{}}\SpecialCharTok{\textbackslash{}\textbackslash{}}\StringTok{url\{....\}\textquotesingle{}}\NormalTok{)}
\NormalTok{\} }\ControlFlowTok{else}\NormalTok{ \{}
\NormalTok{  knitr}\SpecialCharTok{::}\FunctionTok{include\_graphics}\NormalTok{(}\StringTok{"fig/Token\_01.gif"}\NormalTok{)}
\NormalTok{\}}
\end{Highlighting}
\end{Shaded}

\url{....}

\hypertarget{uxcf54uxb4dc-15}{%
\subsection{코드}\label{uxcf54uxb4dc-15}}

\begin{Shaded}
\begin{Highlighting}[]
\NormalTok{move()}
\NormalTok{take()}
\NormalTok{move()}
\NormalTok{put()}
\NormalTok{move()}
\end{Highlighting}
\end{Shaded}

\hypertarget{token-02}{%
\section{토큰 이동 2}\label{token-02}}

\begin{itemize}
\tightlist
\item
  \href{https://reeborg.ca/reeborg.html?lang=ko-en\&mode=python\&menu=worlds\%2Fmenus\%2Freeborg_intro_en.json\&name=Tokens\%202\&url=worlds\%2Ftutorial_en\%2Ftokens2.json}{문제
  바로가기}
\item
  선행 지식

  \begin{itemize}
  \tightlist
  \item
    기본 함수 : \texttt{move()}, \texttt{put()}, \texttt{take()}
  \end{itemize}
\item
  난이도: 1
\item
  \texttt{move\_until\_done()} 함수를 제작해서 \texttt{if}문을 사용하게
  되면 조금더 깔끔하게 목적을 달성할 수 있다.
\end{itemize}

\hypertarget{uxc2e4uxd589uxacb0uxacfc-16}{%
\subsection{실행결과}\label{uxc2e4uxd589uxacb0uxacfc-16}}

\begin{Shaded}
\begin{Highlighting}[]
\ControlFlowTok{if}\NormalTok{ (knitr}\SpecialCharTok{:::}\FunctionTok{is\_latex\_output}\NormalTok{()) \{}
\NormalTok{  knitr}\SpecialCharTok{::}\FunctionTok{asis\_output}\NormalTok{(}\StringTok{\textquotesingle{}}\SpecialCharTok{\textbackslash{}\textbackslash{}}\StringTok{url\{....\}\textquotesingle{}}\NormalTok{)}
\NormalTok{\} }\ControlFlowTok{else}\NormalTok{ \{}
\NormalTok{  knitr}\SpecialCharTok{::}\FunctionTok{include\_graphics}\NormalTok{(}\StringTok{"fig/Token\_02.gif"}\NormalTok{)}
\NormalTok{\}}
\end{Highlighting}
\end{Shaded}

\url{....}

\hypertarget{uxcf54uxb4dc-16}{%
\subsection{코드}\label{uxcf54uxb4dc-16}}

\begin{Shaded}
\begin{Highlighting}[]
\KeywordTok{def}\NormalTok{ move\_until\_done():}
\NormalTok{    move()}
    \ControlFlowTok{if}\NormalTok{ object\_here():}
\NormalTok{        take()}
\NormalTok{        move()}
\NormalTok{        put()}
        
\NormalTok{repeat }\DecValTok{5}\NormalTok{:}
\NormalTok{    move\_until\_done()}
\end{Highlighting}
\end{Shaded}

\hypertarget{token-03}{%
\section{토큰 이동 3}\label{token-03}}

\begin{itemize}
\tightlist
\item
  \href{https://reeborg.ca/reeborg.html?lang=ko-en\&mode=python\&menu=worlds\%2Fmenus\%2Freeborg_intro_en.json\&name=Tokens\%203\&url=worlds\%2Ftutorial_en\%2Ftokens3.json}{문제
  바로가기}
\item
  선행 지식

  \begin{itemize}
  \tightlist
  \item
    기본 함수 : \texttt{move()}, \texttt{put()}, \texttt{take()}
  \item
    테스트 조건: \texttt{object\_here()}, \texttt{at\_goal()}
  \item
    반복과 제어조건: \texttt{while} 루프와 부정(negation)
  \end{itemize}
\item
  난이도: 3
\end{itemize}

\hypertarget{uxc2e4uxd589uxacb0uxacfc-17}{%
\subsection{실행결과}\label{uxc2e4uxd589uxacb0uxacfc-17}}

\begin{Shaded}
\begin{Highlighting}[]
\ControlFlowTok{if}\NormalTok{ (knitr}\SpecialCharTok{:::}\FunctionTok{is\_latex\_output}\NormalTok{()) \{}
\NormalTok{  knitr}\SpecialCharTok{::}\FunctionTok{asis\_output}\NormalTok{(}\StringTok{\textquotesingle{}}\SpecialCharTok{\textbackslash{}\textbackslash{}}\StringTok{url\{....\}\textquotesingle{}}\NormalTok{)}
\NormalTok{\} }\ControlFlowTok{else}\NormalTok{ \{}
\NormalTok{  knitr}\SpecialCharTok{::}\FunctionTok{include\_graphics}\NormalTok{(}\StringTok{"fig/Token\_03.gif"}\NormalTok{)}
\NormalTok{\}}
\end{Highlighting}
\end{Shaded}

\url{....}

\hypertarget{uxcf54uxb4dc-17}{%
\subsection{코드}\label{uxcf54uxb4dc-17}}

\begin{Shaded}
\begin{Highlighting}[]
\KeywordTok{def}\NormalTok{ move\_until\_done():}
\NormalTok{    move()}
    \ControlFlowTok{if}\NormalTok{ object\_here():}
\NormalTok{        take()}
\NormalTok{        move()}
\NormalTok{        put()}
        
\ControlFlowTok{while} \KeywordTok{not}\NormalTok{ at\_goal():}
\NormalTok{    move\_until\_done()}
\end{Highlighting}
\end{Shaded}

\hypertarget{token-04}{%
\section{토큰 이동 4}\label{token-04}}

\begin{itemize}
\tightlist
\item
  \href{https://reeborg.ca/reeborg.html?lang=ko-en\&mode=python\&menu=worlds\%2Fmenus\%2Freeborg_intro_en.json\&name=Tokens\%204\&url=worlds\%2Ftutorial_en\%2Ftokens4.json}{문제
  바로가기}
\item
  선행 지식

  \begin{itemize}
  \tightlist
  \item
    기본 함수 : \texttt{move()}, \texttt{put()}, \texttt{take()}
  \item
    테스트 조건: \texttt{object\_here()}, \texttt{carries\_object()},
    \texttt{at\_goal()}
  \item
    반복과 제어조건: \texttt{while} 루프, \texttt{if} 문과
    부정(negation)
  \end{itemize}
\item
  난이도: 5
\end{itemize}

\hypertarget{uxc2e4uxd589uxacb0uxacfc-18}{%
\subsection{실행결과}\label{uxc2e4uxd589uxacb0uxacfc-18}}

\begin{Shaded}
\begin{Highlighting}[]
\ControlFlowTok{if}\NormalTok{ (knitr}\SpecialCharTok{:::}\FunctionTok{is\_latex\_output}\NormalTok{()) \{}
\NormalTok{  knitr}\SpecialCharTok{::}\FunctionTok{asis\_output}\NormalTok{(}\StringTok{\textquotesingle{}}\SpecialCharTok{\textbackslash{}\textbackslash{}}\StringTok{url\{....\}\textquotesingle{}}\NormalTok{)}
\NormalTok{\} }\ControlFlowTok{else}\NormalTok{ \{}
\NormalTok{  knitr}\SpecialCharTok{::}\FunctionTok{include\_graphics}\NormalTok{(}\StringTok{"fig/Token\_04.gif"}\NormalTok{)}
\NormalTok{\}}
\end{Highlighting}
\end{Shaded}

\url{....}

\hypertarget{uxcf54uxb4dc-18}{%
\subsection{코드}\label{uxcf54uxb4dc-18}}

\begin{Shaded}
\begin{Highlighting}[]
\KeywordTok{def}\NormalTok{ collect\_all():}
    \ControlFlowTok{if}\NormalTok{ object\_here():}
\NormalTok{        take()}
\NormalTok{    move()}

\KeywordTok{def}\NormalTok{ put\_down\_all\_and\_move():}
    \ControlFlowTok{while}\NormalTok{ carries\_object():}
\NormalTok{        put()}
\NormalTok{    move()}
    
\ControlFlowTok{while} \KeywordTok{not}\NormalTok{ at\_goal():}
    \ControlFlowTok{if}\NormalTok{ object\_here():}
\NormalTok{        collect\_all()}
    \ControlFlowTok{elif}\NormalTok{ carries\_object():}
\NormalTok{        put\_down\_all\_and\_move()  }
    \ControlFlowTok{else}\NormalTok{:}
\NormalTok{        move()}
\end{Highlighting}
\end{Shaded}

\hypertarget{token-05}{%
\section{토큰 이동 5}\label{token-05}}

\protect\hyperlink{token-04}{토큰 이동 5}를 일반화하여 토큰이 있는 곳과
그렇지 않는 곳도 리보그가 이동하여 목적을 달성할 수 있도록 한다.

\begin{itemize}
\tightlist
\item
  \href{https://reeborg.ca/reeborg.html?lang=ko-en\&mode=python\&menu=worlds\%2Fmenus\%2Freeborg_intro_en.json\&name=Tokens\%205\&url=worlds\%2Ftutorial_en\%2Ftokens5.json}{문제
  바로가기}
\item
  선행 지식

  \begin{itemize}
  \tightlist
  \item
    기본 함수 : \texttt{move()}, \texttt{put()}, \texttt{take()}
  \item
    테스트 조건: \texttt{object\_here()}, \texttt{carries\_object()},
    \texttt{at\_goal()}
  \item
    반복과 제어조건: \texttt{while} 루프, \texttt{if} 문과
    부정(negation)
  \end{itemize}
\item
  난이도: 5
\end{itemize}

\hypertarget{uxc2e4uxd589uxacb0uxacfc-19}{%
\subsection{실행결과}\label{uxc2e4uxd589uxacb0uxacfc-19}}

\begin{Shaded}
\begin{Highlighting}[]
\ControlFlowTok{if}\NormalTok{ (knitr}\SpecialCharTok{:::}\FunctionTok{is\_latex\_output}\NormalTok{()) \{}
\NormalTok{  knitr}\SpecialCharTok{::}\FunctionTok{asis\_output}\NormalTok{(}\StringTok{\textquotesingle{}}\SpecialCharTok{\textbackslash{}\textbackslash{}}\StringTok{url\{....\}\textquotesingle{}}\NormalTok{)}
\NormalTok{\} }\ControlFlowTok{else}\NormalTok{ \{}
\NormalTok{  knitr}\SpecialCharTok{::}\FunctionTok{include\_graphics}\NormalTok{(}\StringTok{"fig/Token\_05.gif"}\NormalTok{)}
\NormalTok{\}}
\end{Highlighting}
\end{Shaded}

\url{....}

\hypertarget{uxcf54uxb4dc-19}{%
\subsection{코드}\label{uxcf54uxb4dc-19}}

\begin{Shaded}
\begin{Highlighting}[]
\KeywordTok{def}\NormalTok{ collect\_all():}
    \ControlFlowTok{if}\NormalTok{ object\_here():}
\NormalTok{        take()}
\NormalTok{    move()}

\KeywordTok{def}\NormalTok{ put\_down\_all\_and\_move():}
    \ControlFlowTok{while}\NormalTok{ carries\_object():}
\NormalTok{        put()}
\NormalTok{    move()}
    
\ControlFlowTok{while} \KeywordTok{not}\NormalTok{ at\_goal():}
    \ControlFlowTok{if}\NormalTok{ object\_here():}
\NormalTok{        collect\_all()}
    \ControlFlowTok{elif}\NormalTok{ carries\_object():}
\NormalTok{        put\_down\_all\_and\_move()  }
    \ControlFlowTok{else}\NormalTok{:}
\NormalTok{        move()}
\end{Highlighting}
\end{Shaded}

\part{평면 2차원}

\hypertarget{rains}{%
\chapter{비바람 창문 닫기}\label{rains}}

\hypertarget{rain-01}{%
\section{창문 닫기 1}\label{rain-01}}

\begin{itemize}
\tightlist
\item
  \href{https://reeborg.ca/reeborg.html?lang=ko-en\&mode=python\&menu=worlds\%2Fmenus\%2Freeborg_intro_en.json\&name=Rain\%200\&url=worlds\%2Ftutorial_en\%2Frain0.json}{문제
  바로가기}
\item
  선행 지식

  \begin{itemize}
  \tightlist
  \item
    기본 함수 : \texttt{move()}, \texttt{turn\_left()},
    \texttt{build\_wall()}
  \end{itemize}
\item
  난이도: 1
\item
  참고: \href{http://statkclee.github.io/rur-ple/intro/22-rain.htm}{러플
  - 비가 와요}
\end{itemize}

\hypertarget{uxc2e4uxd589uxacb0uxacfc-20}{%
\subsection{실행결과}\label{uxc2e4uxd589uxacb0uxacfc-20}}

\begin{Shaded}
\begin{Highlighting}[]
\ControlFlowTok{if}\NormalTok{ (knitr}\SpecialCharTok{:::}\FunctionTok{is\_latex\_output}\NormalTok{()) \{}
\NormalTok{  knitr}\SpecialCharTok{::}\FunctionTok{asis\_output}\NormalTok{(}\StringTok{\textquotesingle{}}\SpecialCharTok{\textbackslash{}\textbackslash{}}\StringTok{url\{....\}\textquotesingle{}}\NormalTok{)}
\NormalTok{\} }\ControlFlowTok{else}\NormalTok{ \{}
\NormalTok{  knitr}\SpecialCharTok{::}\FunctionTok{include\_graphics}\NormalTok{(}\StringTok{"fig/Rain\_01.gif"}\NormalTok{)}
\NormalTok{\}}
\end{Highlighting}
\end{Shaded}

\url{....}

\hypertarget{uxcf54uxb4dc-20}{%
\subsection{코드}\label{uxcf54uxb4dc-20}}

\begin{Shaded}
\begin{Highlighting}[]
\NormalTok{def }\FunctionTok{turn\_around}\NormalTok{()}\SpecialCharTok{:}
    \FunctionTok{turn\_left}\NormalTok{()    }
    \FunctionTok{turn\_left}\NormalTok{()}

\NormalTok{def }\FunctionTok{close\_window}\NormalTok{()}\SpecialCharTok{:}
    \ControlFlowTok{repeat} \DecValTok{6}\SpecialCharTok{:}
        \FunctionTok{move}\NormalTok{()}
    \FunctionTok{build\_wall}\NormalTok{()    }

\NormalTok{def }\FunctionTok{go\_to\_home}\NormalTok{()}\SpecialCharTok{:}
    \FunctionTok{turn\_around}\NormalTok{()    }
    \ControlFlowTok{repeat} \DecValTok{5}\SpecialCharTok{:}
        \FunctionTok{move}\NormalTok{()}

\FunctionTok{close\_window}\NormalTok{()        }
\FunctionTok{go\_to\_home}\NormalTok{()}
\end{Highlighting}
\end{Shaded}

\hypertarget{rain-02}{%
\section{창문 닫기 2}\label{rain-02}}

\begin{itemize}
\tightlist
\item
  \href{https://reeborg.ca/reeborg.html?lang=ko-en\&mode=python\&menu=worlds\%2Fmenus\%2Freeborg_intro_en.json\&name=Rain\%201\&url=worlds\%2Ftutorial_en\%2Frain1.json}{문제
  바로가기}
\item
  선행 지식

  \begin{itemize}
  \tightlist
  \item
    기본 함수 : \texttt{move()}, \texttt{turn\_left()},
    \texttt{build\_wall()}
  \item
    테스트 조건: \texttt{right\_is\_clear()},
    \texttt{wall\_on\_right()}, \texttt{at\_goal()}
  \item
    반복 및 제어: \texttt{while} 루프, \texttt{if} 문
  \end{itemize}
\item
  난이도: 5
\item
  참고: \href{http://statkclee.github.io/rur-ple/intro/22-rain.htm}{러플
  - 비가 와요}
\end{itemize}

\hypertarget{uxc2e4uxd589uxacb0uxacfc-21}{%
\subsection{실행결과}\label{uxc2e4uxd589uxacb0uxacfc-21}}

\begin{Shaded}
\begin{Highlighting}[]
\ControlFlowTok{if}\NormalTok{ (knitr}\SpecialCharTok{:::}\FunctionTok{is\_latex\_output}\NormalTok{()) \{}
\NormalTok{  knitr}\SpecialCharTok{::}\FunctionTok{asis\_output}\NormalTok{(}\StringTok{\textquotesingle{}}\SpecialCharTok{\textbackslash{}\textbackslash{}}\StringTok{url\{....\}\textquotesingle{}}\NormalTok{)}
\NormalTok{\} }\ControlFlowTok{else}\NormalTok{ \{}
\NormalTok{  knitr}\SpecialCharTok{::}\FunctionTok{include\_graphics}\NormalTok{(}\StringTok{"fig/Rain\_02.gif"}\NormalTok{)}
\NormalTok{\}}
\end{Highlighting}
\end{Shaded}

\url{....}

\hypertarget{uxcf54uxb4dc-21}{%
\subsection{코드}\label{uxcf54uxb4dc-21}}

\begin{Shaded}
\begin{Highlighting}[]
\NormalTok{def }\FunctionTok{turn\_right}\NormalTok{()}\SpecialCharTok{:}
    \FunctionTok{turn\_left}\NormalTok{()}
    \FunctionTok{turn\_left}\NormalTok{()}
    \FunctionTok{turn\_left}\NormalTok{()}

    
\NormalTok{def }\FunctionTok{move\_to\_goal}\NormalTok{()}\SpecialCharTok{:}
    \FunctionTok{move}\NormalTok{()}
    \FunctionTok{turn\_right}\NormalTok{()}

\FunctionTok{move\_to\_goal}\NormalTok{()}
\FunctionTok{move}\NormalTok{() }

\ControlFlowTok{while}\NormalTok{ not }\FunctionTok{at\_goal}\NormalTok{()}\SpecialCharTok{:}
    \ControlFlowTok{if} \FunctionTok{front\_is\_clear}\NormalTok{()}\SpecialCharTok{:}
        \FunctionTok{move}\NormalTok{()}
    \ControlFlowTok{if} \FunctionTok{wall\_in\_front}\NormalTok{()}\SpecialCharTok{:}
        \FunctionTok{turn\_left}\NormalTok{()}
    \ControlFlowTok{if} \FunctionTok{right\_is\_clear}\NormalTok{()}\SpecialCharTok{:}
        \FunctionTok{turn\_right}\NormalTok{()}
        \FunctionTok{build\_wall}\NormalTok{()}
        \FunctionTok{turn\_left}\NormalTok{()}
\end{Highlighting}
\end{Shaded}

\hypertarget{rain-03}{%
\section{창문 닫기 3}\label{rain-03}}

\begin{itemize}
\tightlist
\item
  \href{https://reeborg.ca/reeborg.html?lang=ko-en\&mode=python\&menu=worlds\%2Fmenus\%2Freeborg_intro_en.json\&name=Rain\%202\&url=worlds\%2Ftutorial_en\%2Frain2.json}{문제
  바로가기}
\item
  선행 지식

  \begin{itemize}
  \tightlist
  \item
    기본 함수 : \texttt{move()}, \texttt{turn\_left()},
    \texttt{build\_wall()}
  \item
    테스트 조건: \texttt{right\_is\_clear()},
    \texttt{wall\_on\_right()}, \texttt{at\_goal()}
  \item
    반복 및 제어: \texttt{while} 루프, \texttt{if} 문
  \end{itemize}
\item
  힌트: 우회전할지 창문을 닫을지 리보그를 한번더 이동시키면 2가지 경우가
  존재한다.
\item
  난이도: 8
\item
  참고: \href{http://statkclee.github.io/rur-ple/intro/22-rain.htm}{러플
  - 비가 와요}
\end{itemize}

\hypertarget{uxc2e4uxd589uxacb0uxacfc-22}{%
\subsection{실행결과}\label{uxc2e4uxd589uxacb0uxacfc-22}}

\begin{Shaded}
\begin{Highlighting}[]
\ControlFlowTok{if}\NormalTok{ (knitr}\SpecialCharTok{:::}\FunctionTok{is\_latex\_output}\NormalTok{()) \{}
\NormalTok{  knitr}\SpecialCharTok{::}\FunctionTok{asis\_output}\NormalTok{(}\StringTok{\textquotesingle{}}\SpecialCharTok{\textbackslash{}\textbackslash{}}\StringTok{url\{....\}\textquotesingle{}}\NormalTok{)}
\NormalTok{\} }\ControlFlowTok{else}\NormalTok{ \{}
\NormalTok{  knitr}\SpecialCharTok{::}\FunctionTok{include\_graphics}\NormalTok{(}\StringTok{"fig/Rain\_03.gif"}\NormalTok{)}
\NormalTok{\}}
\end{Highlighting}
\end{Shaded}

\url{....}

\hypertarget{uxcf54uxb4dc-22}{%
\subsection{코드}\label{uxcf54uxb4dc-22}}

\begin{Shaded}
\begin{Highlighting}[]
\NormalTok{def }\FunctionTok{turn\_right}\NormalTok{()}\SpecialCharTok{:}
    \FunctionTok{turn\_left}\NormalTok{()}
    \FunctionTok{turn\_left}\NormalTok{()}
    \FunctionTok{turn\_left}\NormalTok{()}
    
\NormalTok{def }\FunctionTok{move\_to\_goal}\NormalTok{()}\SpecialCharTok{:}
    \ControlFlowTok{repeat} \DecValTok{3}\SpecialCharTok{:}
        \FunctionTok{move}\NormalTok{()}
    \FunctionTok{turn\_right}\NormalTok{()}

\FunctionTok{move\_to\_goal}\NormalTok{()}
\FunctionTok{move}\NormalTok{() }

\NormalTok{def }\FunctionTok{go\_back}\NormalTok{()}\SpecialCharTok{:}
    \FunctionTok{turn\_left}\NormalTok{()}
    \FunctionTok{turn\_left}\NormalTok{()}
    \FunctionTok{move}\NormalTok{()}
    \FunctionTok{turn\_left}\NormalTok{()}
    \FunctionTok{turn\_left}\NormalTok{()}
    
\NormalTok{def }\FunctionTok{close\_window}\NormalTok{()}\SpecialCharTok{:}
    \FunctionTok{turn\_right}\NormalTok{()}
    \FunctionTok{build\_wall}\NormalTok{()}
    \FunctionTok{turn\_left}\NormalTok{()}

\ControlFlowTok{while}\NormalTok{ not }\FunctionTok{at\_goal}\NormalTok{()}\SpecialCharTok{:}
    \ControlFlowTok{if} \FunctionTok{wall\_in\_front}\NormalTok{()}\SpecialCharTok{:}
        \FunctionTok{turn\_left}\NormalTok{()}
    \ControlFlowTok{if} \FunctionTok{front\_is\_clear}\NormalTok{() and }\FunctionTok{right\_is\_clear}\NormalTok{()}\SpecialCharTok{:}    
        \FunctionTok{move}\NormalTok{()}
        \ControlFlowTok{if} \FunctionTok{right\_is\_clear}\NormalTok{()}\SpecialCharTok{:}
            \FunctionTok{go\_back}\NormalTok{()}
            \FunctionTok{turn\_right}\NormalTok{()}
\NormalTok{        elif not }\FunctionTok{right\_is\_clear}\NormalTok{()}\SpecialCharTok{:}    
            \FunctionTok{go\_back}\NormalTok{()            }
            \FunctionTok{close\_window}\NormalTok{()}
    \FunctionTok{move}\NormalTok{()}
\end{Highlighting}
\end{Shaded}

\hypertarget{maze}{%
\chapter{미로 탈출}\label{maze}}

\begin{itemize}
\tightlist
\item
  \href{https://reeborg.ca/reeborg.html?lang=ko-en\&mode=python\&menu=worlds\%2Fmenus\%2Freeborg_intro_en.json\&name=Maze\&url=worlds\%2Ftutorial_en\%2Fmaze1.json}{문제
  바로가기}
\item
  선행 지식

  \begin{itemize}
  \tightlist
  \item
    기본 함수 : \texttt{move()}, \texttt{turn\_left()}
  \item
    테스트 조건: \texttt{front\_is\_clear()},
    \texttt{wall\_in\_front()}, \texttt{right\_is\_clear()},
    \texttt{wall\_on\_right()}, \texttt{at\_goal()}
  \item
    반복 \texttt{while} 루프, \texttt{if}/\texttt{elif}/\texttt{else}
    조건 문
  \end{itemize}
\item
  난이도: 4
\end{itemize}

\hypertarget{uxc2e4uxd589uxacb0uxacfc-23}{%
\subsection{실행결과}\label{uxc2e4uxd589uxacb0uxacfc-23}}

\begin{Shaded}
\begin{Highlighting}[]
\ControlFlowTok{if}\NormalTok{ (knitr}\SpecialCharTok{:::}\FunctionTok{is\_latex\_output}\NormalTok{()) \{}
\NormalTok{  knitr}\SpecialCharTok{::}\FunctionTok{asis\_output}\NormalTok{(}\StringTok{\textquotesingle{}}\SpecialCharTok{\textbackslash{}\textbackslash{}}\StringTok{url\{....\}\textquotesingle{}}\NormalTok{)}
\NormalTok{\} }\ControlFlowTok{else}\NormalTok{ \{}
\NormalTok{  knitr}\SpecialCharTok{::}\FunctionTok{include\_graphics}\NormalTok{(}\StringTok{"fig/Maze.gif"}\NormalTok{)}
\NormalTok{\}}
\end{Highlighting}
\end{Shaded}

\url{....}

\hypertarget{uxcf54uxb4dc-23}{%
\subsection{코드}\label{uxcf54uxb4dc-23}}

\begin{Shaded}
\begin{Highlighting}[]
\KeywordTok{def}\NormalTok{ turn\_right():}
\NormalTok{    turn\_left()}
\NormalTok{    turn\_left()}
\NormalTok{    turn\_left()}
    
\ControlFlowTok{while} \KeywordTok{not}\NormalTok{ at\_goal():}

    \ControlFlowTok{if}\NormalTok{ right\_is\_clear():}
\NormalTok{        turn\_right()}
\NormalTok{        move()}
    \ControlFlowTok{elif}\NormalTok{ front\_is\_clear():}
\NormalTok{        move()}
    \ControlFlowTok{else}\NormalTok{:}
\NormalTok{        turn\_left()    }
\end{Highlighting}
\end{Shaded}

\hypertarget{center}{%
\chapter{중심 (center)}\label{center}}

\hypertarget{center-01}{%
\section{중심 찾기 1}\label{center-01}}

\begin{itemize}
\tightlist
\item
  \href{https://reeborg.ca/reeborg.html?lang=ko-en\&mode=python\&menu=worlds\%2Fmenus\%2Freeborg_intro_en.json\&name=Center\%201\&url=worlds\%2Ftutorial_en\%2Fcenter1.json}{문제
  바로가기}
\item
  선행 지식

  \begin{itemize}
  \tightlist
  \item
    기본 함수 : \texttt{move()}, \texttt{turn\_left()}, \texttt{put()}
  \item
    테스트 조건: \texttt{front\_is\_clear()},
    \texttt{wall\_in\_front()}, \texttt{object\_here()}
  \item
    반복 및 제어: \texttt{while} 루프, \texttt{if} 문
  \item
    난이도: 5
  \end{itemize}
\end{itemize}

\hypertarget{uxc2e4uxd589uxacb0uxacfc-24}{%
\subsection{실행결과}\label{uxc2e4uxd589uxacb0uxacfc-24}}

\begin{Shaded}
\begin{Highlighting}[]
\ControlFlowTok{if}\NormalTok{ (knitr}\SpecialCharTok{:::}\FunctionTok{is\_latex\_output}\NormalTok{()) \{}
\NormalTok{  knitr}\SpecialCharTok{::}\FunctionTok{asis\_output}\NormalTok{(}\StringTok{\textquotesingle{}}\SpecialCharTok{\textbackslash{}\textbackslash{}}\StringTok{url\{....\}\textquotesingle{}}\NormalTok{)}
\NormalTok{\} }\ControlFlowTok{else}\NormalTok{ \{}
\NormalTok{  knitr}\SpecialCharTok{::}\FunctionTok{include\_graphics}\NormalTok{(}\StringTok{"fig/Center\_01.gif"}\NormalTok{)}
\NormalTok{\}}
\end{Highlighting}
\end{Shaded}

\url{....}

\hypertarget{uxcf54uxb4dc-24}{%
\subsection{코드}\label{uxcf54uxb4dc-24}}

\begin{Shaded}
\begin{Highlighting}[]
\NormalTok{def }\FunctionTok{turn\_around}\NormalTok{()}\SpecialCharTok{:}
    \FunctionTok{turn\_left}\NormalTok{()}
    \FunctionTok{turn\_left}\NormalTok{()}
    
\CommentTok{\# 환경설정 {-}{-}{-}{-}{-}{-}{-}{-}{-}{-}{-}{-}{-}{-}{-}{-}{-}{-}    }
\NormalTok{def }\FunctionTok{put\_tokens}\NormalTok{()}\SpecialCharTok{:}
    \DocumentationTok{\#\# 토큰 양 끝에 두기 {-}{-}{-}{-}}
    \FunctionTok{put}\NormalTok{()}
    \ControlFlowTok{while} \FunctionTok{front\_is\_clear}\NormalTok{()}\SpecialCharTok{:}
        \FunctionTok{move}\NormalTok{()}
    \FunctionTok{put}\NormalTok{()}
    \DocumentationTok{\#\# 제자리 돌아오기 {-}{-}{-}{-}{-}}
    \FunctionTok{turn\_around}\NormalTok{()}
    \ControlFlowTok{while} \FunctionTok{front\_is\_clear}\NormalTok{()}\SpecialCharTok{:}
        \FunctionTok{move}\NormalTok{()}
    \FunctionTok{turn\_around}\NormalTok{()}
    
\FunctionTok{put\_tokens}\NormalTok{()    }

\CommentTok{\# 토큰 이동 {-}{-}{-}{-}{-}{-}{-}{-}{-}{-}{-}{-}{-}{-}{-}{-}{-}{-}    }

\NormalTok{def }\FunctionTok{pick\_and\_put}\NormalTok{()}\SpecialCharTok{:}    
    \FunctionTok{take}\NormalTok{()}
    \FunctionTok{move}\NormalTok{()}
    \FunctionTok{put}\NormalTok{()}
    \FunctionTok{move}\NormalTok{()}

\NormalTok{def }\FunctionTok{move\_left\_to\_right}\NormalTok{()}\SpecialCharTok{:}
    \ControlFlowTok{if} \FunctionTok{object\_here}\NormalTok{()}\SpecialCharTok{:}
        \FunctionTok{pick\_and\_put}\NormalTok{()}
        \ControlFlowTok{while}\NormalTok{ not }\FunctionTok{object\_here}\NormalTok{()}\SpecialCharTok{:}
            \FunctionTok{move}\NormalTok{()}

\NormalTok{def }\FunctionTok{move\_right\_to\_left}\NormalTok{()}\SpecialCharTok{:}
    \FunctionTok{turn\_around}\NormalTok{()}
    \ControlFlowTok{if} \FunctionTok{object\_here}\NormalTok{()}\SpecialCharTok{:}
        \FunctionTok{pick\_and\_put}\NormalTok{()}
        \ControlFlowTok{while}\NormalTok{ not }\FunctionTok{object\_here}\NormalTok{()}\SpecialCharTok{:}
            \FunctionTok{move}\NormalTok{()}

\NormalTok{def }\FunctionTok{move\_left\_right\_tokens}\NormalTok{() }\SpecialCharTok{:}
    \FunctionTok{move\_left\_to\_right}\NormalTok{()}
    \FunctionTok{move\_right\_to\_left}\NormalTok{()           }
    \FunctionTok{turn\_around}\NormalTok{()}

  
\FunctionTok{move\_left\_right\_tokens}\NormalTok{()}
\FunctionTok{move\_left\_right\_tokens}\NormalTok{()}
\FunctionTok{move\_left\_right\_tokens}\NormalTok{()}
\FunctionTok{move\_left\_right\_tokens}\NormalTok{()}
\end{Highlighting}
\end{Shaded}

\hypertarget{harvest}{%
\chapter{추수 (Harvest)}\label{harvest}}

\hypertarget{harvest-01}{%
\section{추수 1}\label{harvest-01}}

\begin{itemize}
\tightlist
\item
  \href{https://reeborg.ca/reeborg.html?lang=ko-en\&mode=python\&menu=worlds\%2Fmenus\%2Freeborg_intro_en.json\&name=Harvest\%201\&url=worlds\%2Ftutorial_en\%2Fharvest1.json}{문제
  바로가기}
\item
  선행 지식

  \begin{itemize}
  \tightlist
  \item
    기본 함수 : \texttt{move()}, \texttt{turn\_left()}, \texttt{take()}
  \item
    테스트 함수 : \texttt{object\_here()}
  \item
    반복: \texttt{while} 혹은 \texttt{if} 과 결함
  \end{itemize}
\item
  힌트: \texttt{harvest\_one\_row()} 함수를 작성하고 6번 반복
\item
  난이도: 3
\item
  참고: \href{https://reeborg.ca/docs/ko/variables/harvest3.html}{리보그
  세상 - 추가된 추수 도전과제}
\end{itemize}

\hypertarget{uxc2e4uxd589uxacb0uxacfc-25}{%
\subsection{실행결과}\label{uxc2e4uxd589uxacb0uxacfc-25}}

\begin{Shaded}
\begin{Highlighting}[]
\ControlFlowTok{if}\NormalTok{ (knitr}\SpecialCharTok{:::}\FunctionTok{is\_latex\_output}\NormalTok{()) \{}
\NormalTok{  knitr}\SpecialCharTok{::}\FunctionTok{asis\_output}\NormalTok{(}\StringTok{\textquotesingle{}}\SpecialCharTok{\textbackslash{}\textbackslash{}}\StringTok{url\{....\}\textquotesingle{}}\NormalTok{)}
\NormalTok{\} }\ControlFlowTok{else}\NormalTok{ \{}
\NormalTok{  knitr}\SpecialCharTok{::}\FunctionTok{include\_graphics}\NormalTok{(}\StringTok{"fig/Harvest\_01.gif"}\NormalTok{)}
\NormalTok{\}}
\end{Highlighting}
\end{Shaded}

\url{....}

\hypertarget{uxcf54uxb4dc-25}{%
\subsection{코드}\label{uxcf54uxb4dc-25}}

\begin{Shaded}
\begin{Highlighting}[]
\CommentTok{\# 추수 위치로 이동 {-}{-}{-}{-}}
\NormalTok{def }\FunctionTok{turn\_right}\NormalTok{()}\SpecialCharTok{:}
    \FunctionTok{turn\_left}\NormalTok{()}
    \FunctionTok{turn\_left}\NormalTok{()}
    \FunctionTok{turn\_left}\NormalTok{()}
    
\NormalTok{def }\FunctionTok{ready\_harvest}\NormalTok{()}\SpecialCharTok{:}
    \FunctionTok{move}\NormalTok{()}
    \FunctionTok{turn\_left}\NormalTok{()}
    \FunctionTok{move}\NormalTok{()}
    \FunctionTok{move}\NormalTok{()}
    \FunctionTok{turn\_right}\NormalTok{()}
  
\FunctionTok{ready\_harvest}\NormalTok{()}

\CommentTok{\# 한줄 추수 {-}{-}{-}{-}}
\NormalTok{def }\FunctionTok{harvest\_one\_row}\NormalTok{()}\SpecialCharTok{:}
    \ControlFlowTok{if}\NormalTok{ not }\FunctionTok{object\_here}\NormalTok{()}\SpecialCharTok{:}
        \FunctionTok{move}\NormalTok{()}
    \ControlFlowTok{while} \FunctionTok{object\_here}\NormalTok{()}\SpecialCharTok{:}
        \ControlFlowTok{if} \FunctionTok{object\_here}\NormalTok{()}\SpecialCharTok{:}
            \FunctionTok{take}\NormalTok{()}
        \FunctionTok{move}\NormalTok{()}
          
\CommentTok{\# harvest\_one\_row()}

\CommentTok{\# 다음 추수 위치로 이동}
\NormalTok{def }\FunctionTok{turn\_around}\NormalTok{()}\SpecialCharTok{:}
    \FunctionTok{turn\_left}\NormalTok{()}
    \FunctionTok{turn\_left}\NormalTok{()}

\NormalTok{def }\FunctionTok{move\_to\_next\_row}\NormalTok{()}\SpecialCharTok{:}    
    \FunctionTok{turn\_around}\NormalTok{()}
    \ControlFlowTok{repeat} \DecValTok{7}\SpecialCharTok{:}
        \FunctionTok{move}\NormalTok{()}
    \FunctionTok{turn\_right}\NormalTok{()}
    \FunctionTok{move}\NormalTok{()}
    \FunctionTok{turn\_right}\NormalTok{()}
    
\CommentTok{\# move\_to\_next\_row()}

\CommentTok{\# 전체 6줄 추수}

\ControlFlowTok{repeat} \DecValTok{6}\SpecialCharTok{:}
    \FunctionTok{harvest\_one\_row}\NormalTok{()}
    \FunctionTok{move\_to\_next\_row}\NormalTok{()}
\end{Highlighting}
\end{Shaded}

\hypertarget{harvest-02}{%
\section{추수 2}\label{harvest-02}}

\begin{itemize}
\tightlist
\item
  \href{https://reeborg.ca/reeborg.html?lang=ko-en\&mode=python\&menu=worlds\%2Fmenus\%2Freeborg_intro_en.json\&name=Harvest\%202\&url=worlds\%2Ftutorial_en\%2Fharvest2.json}{문제
  바로가기}
\item
  선행 지식

  \begin{itemize}
  \tightlist
  \item
    기본 함수 : \texttt{move()}, \texttt{turn\_left()}, \texttt{take()}
  \item
    테스트 함수 : \texttt{object\_here()}
  \item
    반복: \texttt{while} 혹은 \texttt{if} 과 결함
  \end{itemize}
\item
  힌트: \texttt{harvest\_one\_row()} 함수를 작성하고 6번 반복
\item
  난이도: 4
\item
  참고: \href{https://reeborg.ca/docs/ko/variables/harvest3.html}{리보그
  세상 - 추가된 추수 도전과제}
\end{itemize}

\hypertarget{uxc2e4uxd589uxacb0uxacfc-26}{%
\subsection{실행결과}\label{uxc2e4uxd589uxacb0uxacfc-26}}

\begin{Shaded}
\begin{Highlighting}[]
\ControlFlowTok{if}\NormalTok{ (knitr}\SpecialCharTok{:::}\FunctionTok{is\_latex\_output}\NormalTok{()) \{}
\NormalTok{  knitr}\SpecialCharTok{::}\FunctionTok{asis\_output}\NormalTok{(}\StringTok{\textquotesingle{}}\SpecialCharTok{\textbackslash{}\textbackslash{}}\StringTok{url\{....\}\textquotesingle{}}\NormalTok{)}
\NormalTok{\} }\ControlFlowTok{else}\NormalTok{ \{}
\NormalTok{  knitr}\SpecialCharTok{::}\FunctionTok{include\_graphics}\NormalTok{(}\StringTok{"fig/Harvest\_02.gif"}\NormalTok{)}
\NormalTok{\}}
\end{Highlighting}
\end{Shaded}

\url{....}

\hypertarget{uxcf54uxb4dc-26}{%
\subsection{코드}\label{uxcf54uxb4dc-26}}

\begin{Shaded}
\begin{Highlighting}[]
\CommentTok{\# 추수 위치로 이동 {-}{-}{-}{-}}
\NormalTok{def }\FunctionTok{turn\_right}\NormalTok{()}\SpecialCharTok{:}
    \FunctionTok{turn\_left}\NormalTok{()}
    \FunctionTok{turn\_left}\NormalTok{()}
    \FunctionTok{turn\_left}\NormalTok{()}
    
\NormalTok{def }\FunctionTok{ready\_harvest}\NormalTok{()}\SpecialCharTok{:}
    \FunctionTok{move}\NormalTok{()}
    \FunctionTok{turn\_left}\NormalTok{()}
    \FunctionTok{move}\NormalTok{()}
    \FunctionTok{move}\NormalTok{()}
    \FunctionTok{turn\_right}\NormalTok{()}
  
\FunctionTok{ready\_harvest}\NormalTok{()}

\CommentTok{\# 한줄 추수 {-}{-}{-}{-}}
\NormalTok{def }\FunctionTok{harvest\_one\_row}\NormalTok{()}\SpecialCharTok{:}
    \ControlFlowTok{repeat} \DecValTok{6}\SpecialCharTok{:}
        \ControlFlowTok{if}\NormalTok{ not }\FunctionTok{object\_here}\NormalTok{()}\SpecialCharTok{:}
            \FunctionTok{move}\NormalTok{()}
        \ControlFlowTok{while} \FunctionTok{object\_here}\NormalTok{()}\SpecialCharTok{:}
            \FunctionTok{take}\NormalTok{()}

\CommentTok{\# 다음 추수 위치로 이동}
\NormalTok{def }\FunctionTok{turn\_around}\NormalTok{()}\SpecialCharTok{:}
    \FunctionTok{turn\_left}\NormalTok{()}
    \FunctionTok{turn\_left}\NormalTok{()}

\NormalTok{def }\FunctionTok{move\_to\_next\_row}\NormalTok{()}\SpecialCharTok{:}    
    \FunctionTok{turn\_around}\NormalTok{()}
    \ControlFlowTok{repeat} \DecValTok{6}\SpecialCharTok{:}
        \FunctionTok{move}\NormalTok{()}
    \FunctionTok{turn\_right}\NormalTok{()}
    \FunctionTok{move}\NormalTok{()}
    \FunctionTok{turn\_right}\NormalTok{()}

\CommentTok{\# 전체 6줄 추수}

\ControlFlowTok{repeat} \DecValTok{6}\SpecialCharTok{:}
    \FunctionTok{harvest\_one\_row}\NormalTok{()}
    \FunctionTok{move\_to\_next\_row}\NormalTok{()}
\end{Highlighting}
\end{Shaded}

\hypertarget{harvest-03}{%
\section{추수 3}\label{harvest-03}}

\begin{itemize}
\tightlist
\item
  \href{https://reeborg.ca/reeborg.html?lang=ko-en\&mode=python\&menu=worlds\%2Fmenus\%2Freeborg_intro_en.json\&name=Harvest\%203\&url=worlds\%2Ftutorial_en\%2Fharvest3.json}{문제
  바로가기}
\item
  선행 지식

  \begin{itemize}
  \tightlist
  \item
    기본 함수 : \texttt{move()}, \texttt{turn\_left()}, \texttt{take()}
  \item
    테스트 함수 : \texttt{object\_here()}
  \item
    반복: \texttt{while} 혹은 \texttt{if} 과 결함
  \end{itemize}
\item
  힌트: \texttt{harvest\_one\_row()} 함수를 작성하고 6번 반복
\item
  난이도: 4
\item
  참고: \href{https://reeborg.ca/docs/ko/variables/harvest3.html}{리보그
  세상 - 추가된 추수 도전과제}
\end{itemize}

\hypertarget{uxc2e4uxd589uxacb0uxacfc-27}{%
\subsection{실행결과}\label{uxc2e4uxd589uxacb0uxacfc-27}}

\begin{Shaded}
\begin{Highlighting}[]
\ControlFlowTok{if}\NormalTok{ (knitr}\SpecialCharTok{:::}\FunctionTok{is\_latex\_output}\NormalTok{()) \{}
\NormalTok{  knitr}\SpecialCharTok{::}\FunctionTok{asis\_output}\NormalTok{(}\StringTok{\textquotesingle{}}\SpecialCharTok{\textbackslash{}\textbackslash{}}\StringTok{url\{....\}\textquotesingle{}}\NormalTok{)}
\NormalTok{\} }\ControlFlowTok{else}\NormalTok{ \{}
\NormalTok{  knitr}\SpecialCharTok{::}\FunctionTok{include\_graphics}\NormalTok{(}\StringTok{"fig/Harvest\_03.gif"}\NormalTok{)}
\NormalTok{\}}
\end{Highlighting}
\end{Shaded}

\url{....}

\hypertarget{uxcf54uxb4dc-27}{%
\subsection{코드}\label{uxcf54uxb4dc-27}}

\begin{Shaded}
\begin{Highlighting}[]
\CommentTok{\# 추수 위치로 이동 {-}{-}{-}{-}}
\NormalTok{def }\FunctionTok{turn\_right}\NormalTok{()}\SpecialCharTok{:}
    \FunctionTok{turn\_left}\NormalTok{()}
    \FunctionTok{turn\_left}\NormalTok{()}
    \FunctionTok{turn\_left}\NormalTok{()}
    
\NormalTok{def }\FunctionTok{ready\_harvest}\NormalTok{()}\SpecialCharTok{:}
    \FunctionTok{move}\NormalTok{()}
    \FunctionTok{move}\NormalTok{()}
    \FunctionTok{turn\_left}\NormalTok{()}
    \FunctionTok{move}\NormalTok{()}
    \FunctionTok{move}\NormalTok{()}
    \FunctionTok{turn\_right}\NormalTok{()}
  
\FunctionTok{ready\_harvest}\NormalTok{()}

\CommentTok{\# 당근 심기 {-}{-}{-}{-}}
\NormalTok{def }\FunctionTok{fix\_one\_row}\NormalTok{()}\SpecialCharTok{:}
    \ControlFlowTok{repeat} \DecValTok{6}\SpecialCharTok{:}
        \ControlFlowTok{while}\NormalTok{ not }\FunctionTok{object\_here}\NormalTok{()}\SpecialCharTok{:}
            \FunctionTok{put}\NormalTok{()}
        \CommentTok{\# 당근 1개 이상인 경우 다 뽑아내고 1개 심는다.            }
        \ControlFlowTok{if} \FunctionTok{object\_here}\NormalTok{()}\SpecialCharTok{:}
            \ControlFlowTok{while} \FunctionTok{object\_here}\NormalTok{()}\SpecialCharTok{:}
                \FunctionTok{take}\NormalTok{()}
            \FunctionTok{put}\NormalTok{()}
            \FunctionTok{move}\NormalTok{()}

\CommentTok{\# 다음 추수 위치로 이동}
\NormalTok{def }\FunctionTok{turn\_around}\NormalTok{()}\SpecialCharTok{:}
    \FunctionTok{turn\_left}\NormalTok{()}
    \FunctionTok{turn\_left}\NormalTok{()}

\NormalTok{def }\FunctionTok{move\_to\_next\_row}\NormalTok{()}\SpecialCharTok{:}    
    \FunctionTok{turn\_around}\NormalTok{()}
    \ControlFlowTok{repeat} \DecValTok{6}\SpecialCharTok{:}
        \FunctionTok{move}\NormalTok{()}
    \FunctionTok{turn\_right}\NormalTok{()}
    \FunctionTok{move}\NormalTok{()}
    \FunctionTok{turn\_right}\NormalTok{()}

\CommentTok{\# 전체 6줄 추수}

\ControlFlowTok{repeat} \DecValTok{6}\SpecialCharTok{:}
    \FunctionTok{fix\_one\_row}\NormalTok{()}
    \FunctionTok{move\_to\_next\_row}\NormalTok{()}
\end{Highlighting}
\end{Shaded}

\hypertarget{storm}{%
\chapter{폭풍 후에 \ldots{}}\label{storm}}

\hypertarget{storm-01}{%
\section{낙엽 청소 1}\label{storm-01}}

\begin{itemize}
\tightlist
\item
  \href{https://reeborg.ca/reeborg.html?lang=ko-en\&mode=python\&menu=worlds\%2Fmenus\%2Freeborg_intro_en.json\&name=Storm\%201\&url=worlds\%2Ftutorial_en\%2Fstorm1.json}{문제
  바로가기}
\item
  선행 지식

  \begin{itemize}
  \tightlist
  \item
    기본 함수 : \texttt{move()}, \texttt{turn\_left()}, \texttt{take()},
    \texttt{toss()}
  \item
    테스트 조건: \texttt{object\_here()}, \texttt{carries\_object()},
    \texttt{front\_is\_clear()}, \texttt{wall\_in\_front()}
  \item
    반복 \texttt{while}문과 제어 조건 \texttt{if} 문을 사용
  \end{itemize}
\item
  난이도: 4
\end{itemize}

\hypertarget{uxc2e4uxd589uxacb0uxacfc-28}{%
\subsection{실행결과}\label{uxc2e4uxd589uxacb0uxacfc-28}}

\begin{Shaded}
\begin{Highlighting}[]
\ControlFlowTok{if}\NormalTok{ (knitr}\SpecialCharTok{:::}\FunctionTok{is\_latex\_output}\NormalTok{()) \{}
\NormalTok{  knitr}\SpecialCharTok{::}\FunctionTok{asis\_output}\NormalTok{(}\StringTok{\textquotesingle{}}\SpecialCharTok{\textbackslash{}\textbackslash{}}\StringTok{url\{....\}\textquotesingle{}}\NormalTok{)}
\NormalTok{\} }\ControlFlowTok{else}\NormalTok{ \{}
\NormalTok{  knitr}\SpecialCharTok{::}\FunctionTok{include\_graphics}\NormalTok{(}\StringTok{"fig/Storm\_01.gif"}\NormalTok{)}
\NormalTok{\}}
\end{Highlighting}
\end{Shaded}

\url{....}

\hypertarget{uxcf54uxb4dc-28}{%
\subsection{코드}\label{uxcf54uxb4dc-28}}

\begin{Shaded}
\begin{Highlighting}[]
\KeywordTok{def}\NormalTok{ turn\_right():}
\NormalTok{    turn\_left()}
\NormalTok{    turn\_left()}
\NormalTok{    turn\_left()}

\KeywordTok{def}\NormalTok{ turn\_around():}
\NormalTok{    turn\_left()}
\NormalTok{    turn\_left()}
    
\ControlFlowTok{while}\NormalTok{ front\_is\_clear():}
\NormalTok{    move()}
    \ControlFlowTok{while}\NormalTok{ object\_here():}
\NormalTok{        take()}

\NormalTok{turn\_around()}
\NormalTok{repeat }\DecValTok{5}\NormalTok{:}
\NormalTok{    move()}

\NormalTok{turn\_right()    }
    
\ControlFlowTok{while}\NormalTok{ carries\_object():}
\NormalTok{    toss() }
\end{Highlighting}
\end{Shaded}

\hypertarget{storm-02}{%
\section{낙엽 청소 2}\label{storm-02}}

\begin{itemize}
\tightlist
\item
  \href{https://reeborg.ca/reeborg.html?lang=ko-en\&mode=python\&menu=worlds\%2Fmenus\%2Freeborg_intro_en.json\&name=Storm\%201\&url=worlds\%2Ftutorial_en\%2Fstorm1.json}{문제
  바로가기}
\item
  선행 지식

  \begin{itemize}
  \tightlist
  \item
    기본 함수 : \texttt{move()}, \texttt{turn\_left()}, \texttt{take()},
    \texttt{toss()}
  \item
    테스트 조건: \texttt{object\_here()}, \texttt{carries\_object()},
    \texttt{front\_is\_clear()}, \texttt{wall\_in\_front()}
  \item
    반복 \texttt{while}문과 제어 조건 \texttt{if} 문을 사용
  \end{itemize}
\item
  난이도: 4
\end{itemize}

\hypertarget{uxc2e4uxd589uxacb0uxacfc-29}{%
\subsection{실행결과}\label{uxc2e4uxd589uxacb0uxacfc-29}}

\begin{Shaded}
\begin{Highlighting}[]
\ControlFlowTok{if}\NormalTok{ (knitr}\SpecialCharTok{:::}\FunctionTok{is\_latex\_output}\NormalTok{()) \{}
\NormalTok{  knitr}\SpecialCharTok{::}\FunctionTok{asis\_output}\NormalTok{(}\StringTok{\textquotesingle{}}\SpecialCharTok{\textbackslash{}\textbackslash{}}\StringTok{url\{....\}\textquotesingle{}}\NormalTok{)}
\NormalTok{\} }\ControlFlowTok{else}\NormalTok{ \{}
\NormalTok{  knitr}\SpecialCharTok{::}\FunctionTok{include\_graphics}\NormalTok{(}\StringTok{"fig/Storm\_02.gif"}\NormalTok{)}
\NormalTok{\}}
\end{Highlighting}
\end{Shaded}

\url{....}

\hypertarget{uxcf54uxb4dc-29}{%
\subsection{코드}\label{uxcf54uxb4dc-29}}

\begin{Shaded}
\begin{Highlighting}[]
\KeywordTok{def}\NormalTok{ turn\_around():}
\NormalTok{    turn\_left()}
\NormalTok{    turn\_left()}

\KeywordTok{def}\NormalTok{ turn\_right():}
\NormalTok{    turn\_left()}
\NormalTok{    turn\_left()}
\NormalTok{    turn\_left()}

\KeywordTok{def}\NormalTok{ collect\_leaves():}
    \ControlFlowTok{while}\NormalTok{ front\_is\_clear():}
        \ControlFlowTok{while}\NormalTok{ object\_here():}
\NormalTok{            take()}
\NormalTok{        move()}
        
\KeywordTok{def}\NormalTok{ first\_collect\_leaves():}
    \ControlFlowTok{while}\NormalTok{ front\_is\_clear():}
        \ControlFlowTok{while}\NormalTok{ object\_here():}
\NormalTok{            take()}
\NormalTok{        move()            }
\NormalTok{    turn\_around()}
\NormalTok{    repeat }\DecValTok{4}\NormalTok{:}
\NormalTok{        move()}
\NormalTok{    turn\_right()}
\NormalTok{    move()}
\NormalTok{    turn\_right()}
\NormalTok{    collect\_leaves()}
\NormalTok{    turn\_left()}
\NormalTok{    move()}
\NormalTok{    turn\_left()}
            
\KeywordTok{def}\NormalTok{ go\_to\_beginning():}
    \ControlFlowTok{while}\NormalTok{ object\_here():}
\NormalTok{        take()    }
\NormalTok{    turn\_around()}
    \ControlFlowTok{while} \KeywordTok{not}\NormalTok{ wall\_in\_front():}
\NormalTok{        move()}
\NormalTok{    turn\_left()}
\NormalTok{    move()}
\NormalTok{    turn\_left()    }

\KeywordTok{def}\NormalTok{ go\_to\_home():}
    \ControlFlowTok{while} \KeywordTok{not}\NormalTok{ wall\_in\_front():}
\NormalTok{        move()}
        \ControlFlowTok{while}\NormalTok{ object\_here():}
\NormalTok{            take()}
\NormalTok{    turn\_left()}
    \ControlFlowTok{while} \KeywordTok{not}\NormalTok{ at\_goal():}
        \ControlFlowTok{if}\NormalTok{ front\_is\_clear():}
\NormalTok{            move()}
        \ControlFlowTok{elif}\NormalTok{ right\_is\_clear():}
\NormalTok{            turn\_right()}
        \ControlFlowTok{elif}\NormalTok{ wall\_in\_front():}
\NormalTok{           turn\_left()}
\NormalTok{    turn\_right()}
    \ControlFlowTok{while}\NormalTok{ carries\_object():}
\NormalTok{        toss()}
        
\NormalTok{move()    }
\NormalTok{first\_collect\_leaves()}
\NormalTok{repeat }\DecValTok{3}\NormalTok{:}
\NormalTok{    collect\_leaves()}
\NormalTok{    go\_to\_beginning()}


\NormalTok{go\_to\_home()}
\end{Highlighting}
\end{Shaded}




\end{document}
